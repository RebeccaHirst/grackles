\documentclass[]{article}
\usepackage{lmodern}
\usepackage{amssymb,amsmath}
\usepackage{ifxetex,ifluatex}
\usepackage{fixltx2e} % provides \textsubscript
\ifnum 0\ifxetex 1\fi\ifluatex 1\fi=0 % if pdftex
  \usepackage[T1]{fontenc}
  \usepackage[utf8]{inputenc}
\else % if luatex or xelatex
  \ifxetex
    \usepackage{mathspec}
  \else
    \usepackage{fontspec}
  \fi
  \defaultfontfeatures{Ligatures=TeX,Scale=MatchLowercase}
\fi
% use upquote if available, for straight quotes in verbatim environments
\IfFileExists{upquote.sty}{\usepackage{upquote}}{}
% use microtype if available
\IfFileExists{microtype.sty}{%
\usepackage{microtype}
\UseMicrotypeSet[protrusion]{basicmath} % disable protrusion for tt fonts
}{}
\usepackage[margin=1in]{geometry}
\usepackage{hyperref}
\hypersetup{unicode=true,
            pdftitle={Does great-tailed grackle space use behavior reflect individual differences in exploration?},
            pdfauthor={McCune KB (University of California Santa Barbara, UCSB, kelseybmccune@gmail.com), Ross C (MPI EVA), Folsom M (Max Planck Institute for Evolutionary Anthropology, MPI EVA), Bergeron L (UCSB), Logan CJ (MPI EVA, corina\_logan@eva.mpg.de)},
            pdfborder={0 0 0},
            breaklinks=true}
\urlstyle{same}  % don't use monospace font for urls
\usepackage{color}
\usepackage{fancyvrb}
\newcommand{\VerbBar}{|}
\newcommand{\VERB}{\Verb[commandchars=\\\{\}]}
\DefineVerbatimEnvironment{Highlighting}{Verbatim}{commandchars=\\\{\}}
% Add ',fontsize=\small' for more characters per line
\usepackage{framed}
\definecolor{shadecolor}{RGB}{248,248,248}
\newenvironment{Shaded}{\begin{snugshade}}{\end{snugshade}}
\newcommand{\KeywordTok}[1]{\textcolor[rgb]{0.13,0.29,0.53}{\textbf{#1}}}
\newcommand{\DataTypeTok}[1]{\textcolor[rgb]{0.13,0.29,0.53}{#1}}
\newcommand{\DecValTok}[1]{\textcolor[rgb]{0.00,0.00,0.81}{#1}}
\newcommand{\BaseNTok}[1]{\textcolor[rgb]{0.00,0.00,0.81}{#1}}
\newcommand{\FloatTok}[1]{\textcolor[rgb]{0.00,0.00,0.81}{#1}}
\newcommand{\ConstantTok}[1]{\textcolor[rgb]{0.00,0.00,0.00}{#1}}
\newcommand{\CharTok}[1]{\textcolor[rgb]{0.31,0.60,0.02}{#1}}
\newcommand{\SpecialCharTok}[1]{\textcolor[rgb]{0.00,0.00,0.00}{#1}}
\newcommand{\StringTok}[1]{\textcolor[rgb]{0.31,0.60,0.02}{#1}}
\newcommand{\VerbatimStringTok}[1]{\textcolor[rgb]{0.31,0.60,0.02}{#1}}
\newcommand{\SpecialStringTok}[1]{\textcolor[rgb]{0.31,0.60,0.02}{#1}}
\newcommand{\ImportTok}[1]{#1}
\newcommand{\CommentTok}[1]{\textcolor[rgb]{0.56,0.35,0.01}{\textit{#1}}}
\newcommand{\DocumentationTok}[1]{\textcolor[rgb]{0.56,0.35,0.01}{\textbf{\textit{#1}}}}
\newcommand{\AnnotationTok}[1]{\textcolor[rgb]{0.56,0.35,0.01}{\textbf{\textit{#1}}}}
\newcommand{\CommentVarTok}[1]{\textcolor[rgb]{0.56,0.35,0.01}{\textbf{\textit{#1}}}}
\newcommand{\OtherTok}[1]{\textcolor[rgb]{0.56,0.35,0.01}{#1}}
\newcommand{\FunctionTok}[1]{\textcolor[rgb]{0.00,0.00,0.00}{#1}}
\newcommand{\VariableTok}[1]{\textcolor[rgb]{0.00,0.00,0.00}{#1}}
\newcommand{\ControlFlowTok}[1]{\textcolor[rgb]{0.13,0.29,0.53}{\textbf{#1}}}
\newcommand{\OperatorTok}[1]{\textcolor[rgb]{0.81,0.36,0.00}{\textbf{#1}}}
\newcommand{\BuiltInTok}[1]{#1}
\newcommand{\ExtensionTok}[1]{#1}
\newcommand{\PreprocessorTok}[1]{\textcolor[rgb]{0.56,0.35,0.01}{\textit{#1}}}
\newcommand{\AttributeTok}[1]{\textcolor[rgb]{0.77,0.63,0.00}{#1}}
\newcommand{\RegionMarkerTok}[1]{#1}
\newcommand{\InformationTok}[1]{\textcolor[rgb]{0.56,0.35,0.01}{\textbf{\textit{#1}}}}
\newcommand{\WarningTok}[1]{\textcolor[rgb]{0.56,0.35,0.01}{\textbf{\textit{#1}}}}
\newcommand{\AlertTok}[1]{\textcolor[rgb]{0.94,0.16,0.16}{#1}}
\newcommand{\ErrorTok}[1]{\textcolor[rgb]{0.64,0.00,0.00}{\textbf{#1}}}
\newcommand{\NormalTok}[1]{#1}
\usepackage{graphicx,grffile}
\makeatletter
\def\maxwidth{\ifdim\Gin@nat@width>\linewidth\linewidth\else\Gin@nat@width\fi}
\def\maxheight{\ifdim\Gin@nat@height>\textheight\textheight\else\Gin@nat@height\fi}
\makeatother
% Scale images if necessary, so that they will not overflow the page
% margins by default, and it is still possible to overwrite the defaults
% using explicit options in \includegraphics[width, height, ...]{}
\setkeys{Gin}{width=\maxwidth,height=\maxheight,keepaspectratio}
\IfFileExists{parskip.sty}{%
\usepackage{parskip}
}{% else
\setlength{\parindent}{0pt}
\setlength{\parskip}{6pt plus 2pt minus 1pt}
}
\setlength{\emergencystretch}{3em}  % prevent overfull lines
\providecommand{\tightlist}{%
  \setlength{\itemsep}{0pt}\setlength{\parskip}{0pt}}
\setcounter{secnumdepth}{0}
% Redefines (sub)paragraphs to behave more like sections
\ifx\paragraph\undefined\else
\let\oldparagraph\paragraph
\renewcommand{\paragraph}[1]{\oldparagraph{#1}\mbox{}}
\fi
\ifx\subparagraph\undefined\else
\let\oldsubparagraph\subparagraph
\renewcommand{\subparagraph}[1]{\oldsubparagraph{#1}\mbox{}}
\fi

%%% Use protect on footnotes to avoid problems with footnotes in titles
\let\rmarkdownfootnote\footnote%
\def\footnote{\protect\rmarkdownfootnote}

%%% Change title format to be more compact
\usepackage{titling}

% Create subtitle command for use in maketitle
\providecommand{\subtitle}[1]{
  \posttitle{
    \begin{center}\large#1\end{center}
    }
}

\setlength{\droptitle}{-2em}

  \title{Does great-tailed grackle space use behavior reflect individual
differences in exploration?}
    \pretitle{\vspace{\droptitle}\centering\huge}
  \posttitle{\par}
    \author{McCune KB (University of California Santa Barbara, UCSB,
\href{mailto:kelseybmccune@gmail.com}{\nolinkurl{kelseybmccune@gmail.com}}),
Ross C (MPI EVA), Folsom M (Max Planck Institute for Evolutionary
Anthropology, MPI EVA), Bergeron L (UCSB),
\href{http://CorinaLogan.com}{Logan CJ} (MPI EVA,
\href{mailto:corina_logan@eva.mpg.de}{\nolinkurl{corina\_logan@eva.mpg.de}})}
    \preauthor{\centering\large\emph}
  \postauthor{\par}
      \predate{\centering\large\emph}
  \postdate{\par}
    \date{2020-01-20}


\begin{document}
\maketitle

\begin{Shaded}
\begin{Highlighting}[]
\CommentTok{#Make code wrap text so it doesn't go off the page when Knitting to PDF}
\KeywordTok{library}\NormalTok{(knitr)}
\NormalTok{opts_chunk}\OperatorTok{$}\KeywordTok{set}\NormalTok{(}\DataTypeTok{tidy.opts=}\KeywordTok{list}\NormalTok{(}\DataTypeTok{width.cutoff=}\DecValTok{60}\NormalTok{),}\DataTypeTok{tidy=}\OtherTok{TRUE}\NormalTok{)}
\end{Highlighting}
\end{Shaded}

\textbf{\emph{Click
\href{https://github.com/corinalogan/grackles/blob/master/README.md}{here}
to navigate to the version-tracked reproducible manuscript (.Rmd file)}}

\subsubsection{ABSTRACT}\label{abstract}

This is one of the first studies planned for our long-term research on
the role of behavioral flexibility in rapid geographic range expansions.
\textbf{Project background:} Behavioral flexibility, the ability to
change behavior when circumstances change based on learning from
previous experience (@mikhalevich\_is\_2017), is thought to play an
important role in a species' ability to successfully adapt to new
environments and expand its geographic range (e.g.,
@lefebvre1997feeding, @griffin2014innovation, @chow2016practice,
@sol2000behavioural, @sol2002behavioural, @sol2005big). However,
behavioral flexibility is rarely directly tested at the individual
level, thus limiting our ability to determine how it relates to other
traits, which limits the power of predictions about a species' ability
to adapt behavior to new environments. We use great-tailed grackles
(Quiscalus mexicanus, a bird species) as a model to investigate this
question because they have rapidly expanded their range into North
America over the past 140 years (@wehtje2003range, @peer2011invasion)
(see an overview of the
\href{https://github.com/corinalogan/grackles/blob/master/README.md}{5-year
project timeline}). \textbf{This investigation:} In this piece of the
long-term project we aim to understand whether experimental measures of
exploration are associated with space use behavior in wild grackles.
Exploration, measured in a separate
\href{http://corinalogan.com/Preregistrations/g_exploration.html}{preregistration},
is interpreted as an individual's response to novelty, such as novel
environments or novel objects (@reale2007integrating), to gather
information that does not satisfy immediate needs
(@mettke2002significance). We will measure the space use behavior of
wild grackles using radio telemetry to find color-banded grackles and
record spatial locations across time using GPS. These results will
inform whether individual differences in space use behavior are
associated with consistent individual differences in exploration, which
could be subject to selection, and thus relate to dispersal behaviors
influencing this species' range expansion within populations.
Furthermore, if space use behavior correlates with experimental measures
of exploration, then space use could be used to inform conservation
management strategies (e.g., which individuals are likely to remain in
new or restored habitat after a translocation (@may2016predicting) in
species where it is not logistically feasible to experimentally measure
exploration. Traditional studies of animal movement behavior require
spatial and temporal independence of data points for statistical
analysis (@swihart1985testing). However, spatial and temporal
autocorrelation (where individuals are found in the same locations
across time, such that subsequent relocations are predictable based on
previous space use) is an intrinsic component of animal behavior and
eliminating it can reduce biological relevance (@dray2010exploratory,
e.g., animal movement behavior is influenced by the available habitat
and resources which are distributed non-randomly across the landscape).
Therefore, in addition to using a typical measure of space use, home
range size, we propose two new methods for analyzing wild grackle
exploratory behaviors. The first will describe individual differences in
movement behavior by analyzing autocorrelation of step length (distance
between two sequential observations) and turning angle for each
individual over time (@pacheco2019nahua), while the second will describe
individual differences in spatial preferences by analyzing the
repeatability of each individual's occurrence in particular geographic
locations.

\subsubsection{A. STATE OF THE DATA}\label{a.-state-of-the-data}

This preregistration uses secondary data: data that are already being
collected for other purposes (GPS points in hypothesis 3 and home range
sizes in prediction 3 in the
\href{http://corinalogan.com/Preregistrations/g_flexforaging.html}{flexibility
and foraging} preregistration). This preregistration was written in June
2019, while at the same time increasing the number of GPS points taken
per time per bird to provide enough data for the analyses here, and
submitted in September 2019 to PCI Ecology for pre-study peer review.

\subsubsection{B. HYPOTHESES}\label{b.-hypotheses}

\paragraph{\texorpdfstring{H1: Individual differences in measures of
exploration using novel environment and novel object tasks (see separate
\href{http://corinalogan.com/Preregistrations/g_exploration.html}{preregistration})
are related to variation in space use (measured via home range size,
autocorrelation of step lengths and turning angles, or whether
individuals are predictably found in the same
locations).}{H1: Individual differences in measures of exploration using novel environment and novel object tasks (see separate preregistration) are related to variation in space use (measured via home range size, autocorrelation of step lengths and turning angles, or whether individuals are predictably found in the same locations).}}\label{h1-individual-differences-in-measures-of-exploration-using-novel-environment-and-novel-object-tasks-see-separate-preregistration-are-related-to-variation-in-space-use-measured-via-home-range-size-autocorrelation-of-step-lengths-and-turning-angles-or-whether-individuals-are-predictably-found-in-the-same-locations.}

\textbf{Prediction 1:} The more exploratory grackles that get closer or
make more touches to the novel object and novel environment will be
found in a larger expanse (home range size), use less predictable
movement patterns (low autocorrelation of step lengths), and occupy a
greater variety of spatial locations. This suggests that exploratory
individuals are more willing to move into novel areas in the wild.

\textbf{Prediction 1 alternative 1:} The more exploratory grackles that
get closer or make more touches to the novel object and novel
environment will use a smaller amount of space (home range size), use
predictable movement patterns (high autocorrelation of step lengths),
and consistently occupy the same spatial locations. This suggests that
more exploratory individuals may be able to more efficiently use habitat
within their home range. For example, in great tits, the slow-exploring
phenotype relates to more in-depth investigatory behaviors towards
changes in the local environment {[}@verbeek1994consistent{]}, switching
to utilization of different resources in the same area
{[}@van2009personality{]}, and better problem-solving abilities
{[}@cole2011personality{]}. Therefore it may not be necessary for these
individuals to move into new areas for resources such as food or mating
opportunities.

\textbf{Prediction 1 alternative 2:} Only performance on the novel
environment task will correlate positively with space use behavior in
the wild. This suggests that perception of, and behavioral interactions
with, novel environments (spatial information) differs from that used
for novel objects {[}@mettke2009spatial{]}.

\textbf{Prediction 1 alternative 3:} Only performance on the novel
object task will correlate positively with space use behavior in the
wild. This suggests that, in these urban populations, space use may
primarily be driven by grackles searching for novel objects that
represent human-provided sources of food.

\textbf{Prediction 1 alternative 4:} There will be no correlation
between an individual's proximity or touches to the novel object or
novel environment and their space use behavior. This suggests that the
experimental measures of exploration either are not relevant enough to
how grackles use space in the wild to be able to measure the same trait,
or they are independent of space use behavior potentially because the
individuals tracked are adults and are already familiar with their home
range and surrounding areas and thus do not need to further explore it
as if it was novel.

\paragraph{H2: Space use behavior will vary among grackles from our
three study populations located along different points in the geographic
range of this species (core, middle of expansion, and range
edge).}\label{h2-space-use-behavior-will-vary-among-grackles-from-our-three-study-populations-located-along-different-points-in-the-geographic-range-of-this-species-core-middle-of-expansion-and-range-edge.}

\textbf{Prediction 2:} Home range sizes will increase, autocorrelation
of step lengths and turning angles will decrease (grackle movement
behavior will be less predictable), and grackles will use a greater
variety of spatial locations as the geographic distance from the
original center of the range increases. Specifically, grackles on the
edge of the range (Northern California), will have larger overall home
range sizes, exhibit more variety in step lengths and turning angles,
and use a greater variety of spatial locations than grackles in the core
of the range (Central America). Grackles in the middle of the expanded
range will be intermediate in space use (Arizona). This suggests that
range expansion likely occurs as a result of individual differences in
space use: the individuals on the leading edge of the range expansion
show more exploratory movement behavior {[}@duckworth2007coupling{]}.

\textbf{Prediction 2 alternative 1:} Grackles on the edge of the range
will have smaller overall home range sizes, high autocorrelation in step
length and turning angle (movement behavior will be more predictable),
and consistently use the same spatial locations compared to grackles in
the middle or core of the current range. This suggests that suitable
habitat may be distributed in small patches, there may be higher
predation on grackles that use more space, and/or individual grackles
specialize on certain novel habitat types that are patchily distributed.

\textbf{Prediction 2 alternative 2:} There is no difference across the
geographic range in the space use behavior of grackles. This suggests
that, on average, all grackles use the same amount of space, or that
there is a similar distribution of individual differences in space use
in each population.

\subsubsection{C. METHODS}\label{c.-methods}

\paragraph{\texorpdfstring{\textbf{Planned
Sample}}{Planned Sample}}\label{planned-sample}

Great-tailed grackles are caught in the wild and given colored leg bands
in unique combinations for individual identification. Some individuals
(\textasciitilde{}60) are brought temporarily into aviaries where we
experimentally measure exploration, and then they are released back to
the wild. Grackles are individually housed in an aviary (each 244cm long
by 122cm wide by 213cm tall) for a maximum of six months where they have
ad lib access to water. Grackles are fed Mazuri Small Bird maintenance
diet ad lib during non-testing hours (minimum 20h per day), and various
other food items (e.g., peanuts, grapes, crackers) during testing (up to
4h per day per bird). Individuals are given three to four days to
habituate to the aviaries and then their test battery begins on the
fourth or fifth day (birds are usually tested six days per week,
therefore if their fourth day in the aviaries occurs on a day off, then
they are tested on the fifth day instead). Before release, all grackles
are fitted with VHF radio tags (model A2455 by Advanced Telemetry
Systems or model BD-2 by Holohil Systems Ltd.) so we can track space use
behavior using radio telemetry. Radio tags are attached to the grackles
by gluing them to their backs (@johnson2001great, @mong2007optimizing)
or by using a leg loop harness (methods as in @rappole1991new) made from
sutures (Vicryl undyed 36in sutures, item number D9389 at eSutures.com;
0.5mm diameter, absorbable so they fall off after a few months).

After release, an experimenter tracks each tagged grackle for
approximately 1.5 hours on a given day, taking a GPS point every 1
minute or every 10 minutes (we will determine the sampling interval that
gives the best resolution), regardless of whether the bird moved
{[}@cushman2005elephants{]}. Researchers are careful not to get too
close so as not to influence the grackle's behavior.

\paragraph{\texorpdfstring{\textbf{Sample size
rationale}}{Sample size rationale}}\label{sample-size-rationale}

We test as many birds as we can in the approximately five years of this
study given that the birds are only brought into the aviaries during the
non-breeding season (approximately September through March). The minimum
sample size for captive subjects will be 57 across the three sites. We
catch grackles with a variety of methods, some of which decrease the
likelihood of a selection bias for exploratory and bold individuals
because grackles cannot see the traps (i.e., mist nets). Once released,
we primarily track the space use behavior of the grackles that were in
the aviary, but we also collect GPS point locations on all occasions
that we see any color-marked bird so we can determine whether grackles
that were previously in the aviary have different space use behavior
from non-aviary-held grackles after their release.

\paragraph{\texorpdfstring{\textbf{Data collection stopping
rule}}{Data collection stopping rule}}\label{data-collection-stopping-rule}

We will stop collecting GPS location data on tagged birds when home
ranges are fully revealed. To determine at what point home ranges have
been fully revealed, we will calculate the asymptotic convergence of
home range area as in @leo2016home. We will test home range asymptotic
convergence for breeding season and non-breeding season movements
separately (breeding season: Apr - Aug, non-breeding season: Sep - Mar).

\paragraph{\texorpdfstring{\textbf{Open
materials}}{Open materials}}\label{open-materials}

Testing protocols:

\begin{itemize}
\tightlist
\item
  \href{https://docs.google.com/document/d/1sEMc5z2fw6S9C-wVfc2zV331CRPpu3NuA7IhSFUZJpE/edit?usp=sharing}{Exploration
  protocol} for exploration of new environments and objects, boldness,
  persistence, and motor diversity.
\item
  \href{https://docs.google.com/document/d/1jtjgeWJoZ0Q1CfUpV6zdkyQL3p3WfW9KgyLrMNmNMJc/edit?usp=sharing}{Radio
  tracking protocol} for attaching radio tags and collecting GPS points
  using radio telemetry.
\end{itemize}

\paragraph{\texorpdfstring{\textbf{Open
data}}{Open data}}\label{open-data}

When the study is complete, the data will be published in the Knowledge
Network for Biocomplexity's data repository.

\paragraph{\texorpdfstring{\textbf{Randomization and
counterbalancing}}{Randomization and counterbalancing}}\label{randomization-and-counterbalancing}

There is no randomization in this investigation. The order of the
exploration tasks is counterbalanced across birds as in this
\href{http://corinalogan.com/Preregistrations/g_exploration.html}{separate
preregistration}. The time of day that we collect GPS point locations is
counterbalanced within and across birds to account for variation in
movement behavior arising from daily circadian rhythms.

\paragraph{\texorpdfstring{\textbf{Blinding of conditions during
analysis}}{Blinding of conditions during analysis}}\label{blinding-of-conditions-during-analysis}

No blinding is involved in this investigation.

\paragraph{\texorpdfstring{\textbf{Dependent
variables}}{Dependent variables}}\label{dependent-variables}

\textbf{P1-P2}

\begin{enumerate}
\def\labelenumi{\arabic{enumi})}
\item
  Home range size (square meters): an estimate calculated with the
  minimum convex polygon technique, which consists of the smallest
  polygon to enclose all GPS location points for an individual grackle
  during its normal activities {[}@calenge2011home{]}. Because we are
  interested in the exploratory movements of grackles, we will not
  exclude any outlier GPS locations.
\item
  Autocorrelation of step length (meters): measured as the standard
  deviation of step lengths (the distance between two sequential GPS
  points)
\item
  Autocorrelation of turning angle (degrees): measured as the standard
  deviation of turning angles
\item
  Spatial location preference: measured as the repeatability of grackle
  occurrence in a given cell of a 5 x 5m grid array across the landscape
\end{enumerate}

One model will be run for each dependent variable

\paragraph{\texorpdfstring{\textbf{Independent
variables}}{Independent variables}}\label{independent-variables}

\textbf{P1 and P1 alternatives 2-4}

\begin{enumerate}
\def\labelenumi{\arabic{enumi})}
\item
  Exploration of novel environment: Latency to approach to 20cm of a
  novel environment (that does not contain food) set inside a familiar
  environment (that contains maintenance diet away from the object) - OR
  - closest approach distance to the novel environment (choose the
  variable with the most data)
\item
  Exploration of novel object: Latency to approach to 20cm of an object
  (novel or familiar, that does not contain food) in a familiar
  environment (that contains maintenance diet away from the object) - OR
  - closest approach distance to the object (choose the variable with
  the most data)
\item
  Sex: Male or female
\item
  Condition: Whether the grackle was only ever in the wild or whether it
  had been temporarily held in aviaries before data collection on space
  use
\end{enumerate}

\textbf{P1 alternative 1}

\begin{enumerate}
\def\labelenumi{\arabic{enumi})}
\item
  Exploration of novel environment: Latency to approach to 20cm of a
  novel environment (that does not contain food) set inside a familiar
  environment (that contains maintenance diet away from the object) - OR
  - closest approach distance to the novel environment (choose the
  variable with the most data)
\item
  Exploration of novel object: Latency to approach to 20cm of an object
  (novel or familiar, that does not contain food) in a familiar
  environment (that contains maintenance diet away from the object) - OR
  - closest approach distance to the object (choose the variable with
  the most data)
\item
  Problem-solving performance: Measured as performance on multi-access
  box tasks described in this
  \href{http://corinalogan.com/Preregistrations/g_flexmanip.html}{separate
  preregistration}
\item
  Number of different foods eaten and foraging techniques used: Measured
  during focal follow observations on wild grackles, described in this
  \href{http://corinalogan.com/Preregistrations/g_flexforaging.html}{separate
  preregistration}
\item
  Sex: Male or female
\item
  Condition: Whether the grackle was only ever in the wild or whether it
  had been temporarily held in aviaries before data collection on space
  use
\end{enumerate}

\textbf{P2}

\begin{enumerate}
\def\labelenumi{\arabic{enumi})}
\item
  Site: Whether the grackle was from our study population located on the
  edge of the range (Northern California), the center of the original
  range (Central America), or the center of the current expanded range
  (Arizona).
\item
  Sex: Male or female
\item
  Condition: Whether the grackle was only ever in the wild or whether it
  had been temporarily held in aviaries before data collection on space
  use
\end{enumerate}

\subsubsection{D. ANALYSIS PLAN}\label{d.-analysis-plan}

We do not plan to \textbf{exclude} any data and no data are
\textbf{missing}. Analyses will be conducted in R (current version
3.5.2; @rcoreteam) and Stan (version 2.18, @carpenter2017stan).

We will first verify that the GPS point locations on each bird result in
asymptotic convergence as in @leo2016home. To calculate our dependent
variables we will use the minimum convex polygon method for estimating
home range size (in square meters) using the R packages adehabitatHR
{[}@calenge2006package{]} and sf {[}@pebesma2018simple{]}. Minimum
convex polygons are widely used to estimate home range size and they
only include points within the areas used by the animal during ``normal
activities'' {[}@calenge2011home{]}. Therefore, outlier relocation
detections are usually excluded. We are interested in all exploratory
movements by grackles, so we will not exclude any outlier relocations.

From the GPS point locations collected on each individual, we will use a
Bayesian model (detailed below) to estimate the following parameters:
mean and dispersion (variance) of step lengths and turning angles for
each bird on each daily track (@pacheco2019nahua). We will determine
whether these parameters governing movement are stable (or variable)
within individuals across days. A small variance would indicate there is
low variability (high repeatability) in the daily movement behaviors of
the individual.

Moreover, we will determine whether grackles show individual differences
in consistent use of habitat by overlaying a grid array across the
landscape. We will then create matrices describing the number of times a
grackle was observed in each cell on each day. High autocorrelation
among daily matrices indicates a low-exploring individual that frequents
the same spatial locations across days.

We will then model the relationship between bird-specific data on
performance at our exploration tasks (and other covariates), and
bird-specific movement parameters (e.g.~step-size, turning angle,
autocorrelation in space use).

\paragraph{\texorpdfstring{\emph{Calculating home range
size}}{Calculating home range size}}\label{calculating-home-range-size}

\begin{Shaded}
\begin{Highlighting}[]
\CommentTok{#Load packages}
\KeywordTok{library}\NormalTok{(adehabitatHR)}
\KeywordTok{library}\NormalTok{(sf)}

\CommentTok{#Point to the correct data file and load it}
\KeywordTok{setwd}\NormalTok{(}\StringTok{"~/Documents/Grackle project/Space use"}\NormalTok{)}

\NormalTok{pts<-}\KeywordTok{read.csv}\NormalTok{(}\StringTok{"gtgr_points.csv"}\NormalTok{, }\DataTypeTok{header =}\NormalTok{ T)}

\CommentTok{#Set variables to ensure the models read the data properly}
\NormalTok{pts}\OperatorTok{$}\NormalTok{Latitude =}\StringTok{ }\KeywordTok{as.numeric}\NormalTok{(}\KeywordTok{as.character}\NormalTok{(pts}\OperatorTok{$}\NormalTok{Latitude))}
\NormalTok{pts}\OperatorTok{$}\NormalTok{Longitude =}\StringTok{ }\KeywordTok{as.numeric}\NormalTok{(}\KeywordTok{as.character}\NormalTok{(pts}\OperatorTok{$}\NormalTok{Longitude))}
\NormalTok{pts}\OperatorTok{$}\NormalTok{Date =}\StringTok{ }\KeywordTok{as.character}\NormalTok{(pts}\OperatorTok{$}\NormalTok{Date)}
\NormalTok{pts}\OperatorTok{$}\NormalTok{Time =}\StringTok{ }\KeywordTok{as.character}\NormalTok{(pts}\OperatorTok{$}\NormalTok{Time)}
\NormalTok{pts}\OperatorTok{$}\NormalTok{da.ti =}\StringTok{ }\KeywordTok{as.POSIXct}\NormalTok{(}\KeywordTok{paste}\NormalTok{(pts}\OperatorTok{$}\NormalTok{Date, pts}\OperatorTok{$}\NormalTok{Time), }\DataTypeTok{format=}\StringTok{"%Y-%m-%d %H:%M:%S"}\NormalTok{)}

\NormalTok{### Home range size can only be calculated when a bird has more than 5 relocations}
\NormalTok{### Count the number of relocations for each bird, then exclude individuals with less than 6}
\NormalTok{tmp =}\StringTok{ }\NormalTok{pts}
\NormalTok{tmp}\OperatorTok{$}\NormalTok{count =}\StringTok{ }\DecValTok{1}
\NormalTok{tmp =}\StringTok{ }\KeywordTok{aggregate}\NormalTok{(count }\OperatorTok{~}\StringTok{ }\NormalTok{Bird.Name, }\DataTypeTok{FUN =} \StringTok{"sum"}\NormalTok{, }\DataTypeTok{data =}\NormalTok{ tmp)}
\NormalTok{pts.min =}\StringTok{ }\KeywordTok{merge}\NormalTok{(pts, tmp, }\DataTypeTok{by =} \StringTok{"Bird.Name"}\NormalTok{, }\DataTypeTok{all =}\NormalTok{ T)}
\NormalTok{pts.min =}\StringTok{ }\NormalTok{pts.min[}\OperatorTok{-}\KeywordTok{which}\NormalTok{(pts.min}\OperatorTok{$}\NormalTok{count}\OperatorTok{<}\DecValTok{6}\NormalTok{),]}
\NormalTok{pts.min}\OperatorTok{$}\NormalTok{Bird.Name =}\StringTok{ }\KeywordTok{as.factor}\NormalTok{(}\KeywordTok{as.character}\NormalTok{(pts.min}\OperatorTok{$}\NormalTok{Bird.Name))}

\NormalTok{### Convert dataframe to Spatial file type}
\NormalTok{p.sf <-}\StringTok{ }\KeywordTok{st_as_sf}\NormalTok{(pts.min, }\DataTypeTok{coords =} \KeywordTok{c}\NormalTok{(}\StringTok{"Longitude"}\NormalTok{, }\StringTok{"Latitude"}\NormalTok{), }\DataTypeTok{crs =} \DecValTok{4326}\NormalTok{) }
\KeywordTok{class}\NormalTok{(p.sf) }
\NormalTok{p.spatial <-}\StringTok{ }\KeywordTok{as}\NormalTok{(p.sf, }\StringTok{"Spatial"}\NormalTok{)}
\KeywordTok{class}\NormalTok{(p.spatial)}

\NormalTok{### Calculate home range in square meters for each individual, excluding no outliers (percent = 100)}
\NormalTok{hr =}\StringTok{ }\KeywordTok{mcp}\NormalTok{(p.spatial[}\DecValTok{1}\NormalTok{], }\DataTypeTok{percent =} \DecValTok{100}\NormalTok{, }\DataTypeTok{unout =} \StringTok{"m2"}\NormalTok{)}
\KeywordTok{plot}\NormalTok{(hr)}
\NormalTok{hr =}\StringTok{ }\KeywordTok{as.data.frame}\NormalTok{(hr)}

\NormalTok{### Normalize data for regression model }
\KeywordTok{hist}\NormalTok{(}\KeywordTok{log}\NormalTok{(hr}\OperatorTok{$}\NormalTok{area)) }

\NormalTok{### Make data sheet with the experimental exploration measures combined with the home range measures}
\NormalTok{exp <-}\StringTok{ }\KeywordTok{read.csv}\NormalTok{(}\StringTok{"Explore_combined.csv"}\NormalTok{, }\DataTypeTok{header =}\NormalTok{ T)}
\NormalTok{hr_exp =}\StringTok{ }\KeywordTok{merge}\NormalTok{(exp, hr, }\DataTypeTok{by =} \StringTok{"id"}\NormalTok{, }\DataTypeTok{all =}\NormalTok{ T)}
\KeywordTok{write.csv}\NormalTok{(hr_exp, }\StringTok{"space_use.csv"}\NormalTok{)}
\end{Highlighting}
\end{Shaded}

\paragraph{\texorpdfstring{\emph{Code to create functions for analyzing
movement
behaviors}}{Code to create functions for analyzing movement behaviors}}\label{code-to-create-functions-for-analyzing-movement-behaviors}

All scripts and code are available at
\url{https://github.com/ctross/grackleator}.

\begin{Shaded}
\begin{Highlighting}[]
\NormalTok{### gracklenomial.R}
\NormalTok{gracklenomial <-}\StringTok{ }\ControlFlowTok{function}\NormalTok{(GrackBins)\{}
\NormalTok{model_dat_}\DecValTok{2}\NormalTok{=}\KeywordTok{list}\NormalTok{(}
\DataTypeTok{Trips=}\KeywordTok{dim}\NormalTok{(GrackBins)[}\DecValTok{1}\NormalTok{],}
\DataTypeTok{Bins=}\KeywordTok{dim}\NormalTok{(GrackBins)[}\DecValTok{2}\NormalTok{],}
\DataTypeTok{GrackBins=}\NormalTok{GrackBins)}

\NormalTok{model_code_}\DecValTok{2}\NormalTok{ <-}\StringTok{ '}
\StringTok{data\{}
\StringTok{    int Trips;}
\StringTok{    int Bins;}
\StringTok{    int GrackBins [Trips,Bins];}
\StringTok{\}}

\StringTok{parameters\{}
\StringTok{ simplex[Bins] A;}
\StringTok{ real<lower=0,upper=1> B;}
\StringTok{\}}

\StringTok{model\{}
\StringTok{ vector[Bins] C;}
\StringTok{ A ~ dirichlet(rep_vector(1,Bins));}
\StringTok{ B ~ beta(1,1);}

\StringTok{ for( i in 2:Trips)\{}
\StringTok{    C = to_vector(GrackBins[i-1])/sum(GrackBins[i-1]);}
\StringTok{    GrackBins[i] ~ multinomial( A*(1-B) + B*C  );}
\StringTok{    \}}
\StringTok{\}}
\StringTok{'}
\NormalTok{ m2 <-}\StringTok{ }\KeywordTok{stan}\NormalTok{( }\DataTypeTok{model_code=}\NormalTok{model_code_}\DecValTok{2}\NormalTok{, }\DataTypeTok{data=}\NormalTok{model_dat_}\DecValTok{2}\NormalTok{,}\DataTypeTok{refresh=}\DecValTok{1}\NormalTok{,}\DataTypeTok{chains=}\DecValTok{1}\NormalTok{, }\DataTypeTok{control =} \KeywordTok{list}\NormalTok{(}\DataTypeTok{adapt_delta =} \FloatTok{0.9}\NormalTok{, }\DataTypeTok{max_treedepth =} \DecValTok{13}\NormalTok{))}

 \KeywordTok{return}\NormalTok{(m2)}
\NormalTok{\}}


\NormalTok{### gracklepars.R}
\NormalTok{gracklepars <-}\StringTok{ }\ControlFlowTok{function}\NormalTok{(m1)\{}
\NormalTok{MuD_M<-}\KeywordTok{data.frame}\NormalTok{(}\DataTypeTok{Group=}\StringTok{"Step-Size"}\NormalTok{,}\DataTypeTok{Group2=}\StringTok{"Mean"}\NormalTok{,}\DataTypeTok{Group3=}\StringTok{"Mean"}\NormalTok{,}\DataTypeTok{Value=}\NormalTok{rstan}\OperatorTok{::}\KeywordTok{extract}\NormalTok{(m1,}\DataTypeTok{pars=}\StringTok{"MuD_M"}\NormalTok{)}\OperatorTok{$}\NormalTok{MuD_M) }\CommentTok{# Mean Step-Size over Days}
\NormalTok{DispD_M<-}\KeywordTok{data.frame}\NormalTok{(}\DataTypeTok{Group=}\StringTok{"Step-Size"}\NormalTok{,}\DataTypeTok{Group2=}\StringTok{"Mean"}\NormalTok{,}\DataTypeTok{Group3=}\StringTok{"Dispersion"}\NormalTok{,}\DataTypeTok{Value=}\NormalTok{rstan}\OperatorTok{::}\KeywordTok{extract}\NormalTok{(m1,}\DataTypeTok{pars=}\StringTok{"DispD_M"}\NormalTok{)}\OperatorTok{$}\NormalTok{DispD_M) }\CommentTok{# Dispersion in Mean Step-Size over days}

\NormalTok{MuD_D <-}\KeywordTok{data.frame}\NormalTok{(}\DataTypeTok{Group=}\StringTok{"Step-Size"}\NormalTok{,}\DataTypeTok{Group2=}\StringTok{"Dispersion"}\NormalTok{,}\DataTypeTok{Group3=}\StringTok{"Mean"}\NormalTok{,}\DataTypeTok{Value=}\NormalTok{rstan}\OperatorTok{::}\KeywordTok{extract}\NormalTok{(m1,}\DataTypeTok{pars=}\StringTok{"MuD_D"}\NormalTok{)}\OperatorTok{$}\NormalTok{MuD_D)}\CommentTok{#  Mean Dispersion in Step-Size }
\NormalTok{DispD_D <-}\KeywordTok{data.frame}\NormalTok{(}\DataTypeTok{Group=}\StringTok{"Step-Size"}\NormalTok{,}\DataTypeTok{Group2=}\StringTok{"Dispersion"}\NormalTok{,}\DataTypeTok{Group3=}\StringTok{"Dispersion"}\NormalTok{,}\DataTypeTok{Value=}\NormalTok{rstan}\OperatorTok{::}\KeywordTok{extract}\NormalTok{(m1,}\DataTypeTok{pars=}\StringTok{"DispD_D"}\NormalTok{)}\OperatorTok{$}\NormalTok{DispD_D)}\CommentTok{# Dispersion in Dispersion in Step-Size }

\NormalTok{MuA_M <-}\KeywordTok{data.frame}\NormalTok{(}\DataTypeTok{Group=}\StringTok{"Heading Change"}\NormalTok{,}\DataTypeTok{Group2=}\StringTok{"Mean"}\NormalTok{,}\DataTypeTok{Group3=}\StringTok{"Mean"}\NormalTok{,}\DataTypeTok{Value=}\NormalTok{rstan}\OperatorTok{::}\KeywordTok{extract}\NormalTok{(m1,}\DataTypeTok{pars=}\StringTok{"MuA_M"}\NormalTok{)}\OperatorTok{$}\NormalTok{MuA_M)}\CommentTok{# Mean Angle Change over Days}
\NormalTok{DispA_M <-}\KeywordTok{data.frame}\NormalTok{(}\DataTypeTok{Group=}\StringTok{"Heading Change"}\NormalTok{,}\DataTypeTok{Group2=}\StringTok{"Mean"}\NormalTok{,}\DataTypeTok{Group3=}\StringTok{"Dispersion"}\NormalTok{,}\DataTypeTok{Value=}\NormalTok{rstan}\OperatorTok{::}\KeywordTok{extract}\NormalTok{(m1,}\DataTypeTok{pars=}\StringTok{"DispA_M"}\NormalTok{)}\OperatorTok{$}\NormalTok{DispA_M)}\CommentTok{# Dispersion in Mean Angle Change over days}

\NormalTok{MuA_D <-}\KeywordTok{data.frame}\NormalTok{(}\DataTypeTok{Group=}\StringTok{"Heading Change"}\NormalTok{,}\DataTypeTok{Group2=}\StringTok{"Dispersion"}\NormalTok{,}\DataTypeTok{Group3=}\StringTok{"Mean"}\NormalTok{,}\DataTypeTok{Value=}\NormalTok{rstan}\OperatorTok{::}\KeywordTok{extract}\NormalTok{(m1,}\DataTypeTok{pars=}\StringTok{"MuA_D"}\NormalTok{)}\OperatorTok{$}\NormalTok{MuA_D)}\CommentTok{#  Mean Dispersion in Angle Change }
\NormalTok{DispA_D <-}\KeywordTok{data.frame}\NormalTok{(}\DataTypeTok{Group=}\StringTok{"Heading Change"}\NormalTok{,}\DataTypeTok{Group2=}\StringTok{"Dispersion"}\NormalTok{,}\DataTypeTok{Group3=}\StringTok{"Dispersion"}\NormalTok{,}\DataTypeTok{Value=}\NormalTok{rstan}\OperatorTok{::}\KeywordTok{extract}\NormalTok{(m1,}\DataTypeTok{pars=}\StringTok{"DispA_D"}\NormalTok{)}\OperatorTok{$}\NormalTok{DispA_D)}\CommentTok{# Dispersion in Dispersion in Angle Change}

\NormalTok{df_allx<-}\KeywordTok{rbind}\NormalTok{(MuD_M,DispD_M,MuD_D,DispD_D,MuA_M,DispA_M,MuA_D,DispA_D)}

 \KeywordTok{ggplot}\NormalTok{(df_allx, }\KeywordTok{aes}\NormalTok{(}\DataTypeTok{x=}\NormalTok{Value)) }\OperatorTok{+}\StringTok{ }\KeywordTok{geom_density}\NormalTok{(}\KeywordTok{aes}\NormalTok{(}\DataTypeTok{group=}\NormalTok{Group3, }\DataTypeTok{colour=}\NormalTok{Group3, }\DataTypeTok{fill=}\NormalTok{Group3), }\DataTypeTok{alpha=}\FloatTok{0.3}\NormalTok{)   }\OperatorTok{+}\StringTok{ }\KeywordTok{facet_wrap}\NormalTok{(Group2}\OperatorTok{~}\NormalTok{Group,}\DataTypeTok{scales=}\StringTok{"free"}\NormalTok{)}\OperatorTok{+}
\StringTok{ }\KeywordTok{labs}\NormalTok{(}\DataTypeTok{y=}\StringTok{"Regression parameters"}\NormalTok{) }\OperatorTok{+}\StringTok{ }\KeywordTok{theme}\NormalTok{(}\DataTypeTok{strip.text.x =} \KeywordTok{element_text}\NormalTok{(}\DataTypeTok{size=}\DecValTok{14}\NormalTok{,}\DataTypeTok{face=}\StringTok{"bold"}\NormalTok{),}\DataTypeTok{axis.text=}\KeywordTok{element_text}\NormalTok{(}\DataTypeTok{size=}\DecValTok{12}\NormalTok{),}
        \DataTypeTok{axis.title=}\KeywordTok{element_text}\NormalTok{(}\DataTypeTok{size=}\DecValTok{14}\NormalTok{,}\DataTypeTok{face=}\StringTok{"bold"}\NormalTok{)) }
\NormalTok{\}}


\NormalTok{### grackletrip.R}
\NormalTok{grackletrip <-}\StringTok{ }\ControlFlowTok{function}\NormalTok{(m1)\{}
\NormalTok{ m1a<-rstan}\OperatorTok{::}\KeywordTok{extract}\NormalTok{(m1,}\DataTypeTok{pars=}\StringTok{"AlphaAngle"}\NormalTok{)}
\NormalTok{ sample_eff<-}\KeywordTok{apply}\NormalTok{(m1a}\OperatorTok{$}\NormalTok{AlphaAngle,}\DecValTok{2}\NormalTok{,quantile,}\DataTypeTok{probs=}\KeywordTok{c}\NormalTok{(}\FloatTok{0.05}\NormalTok{,}\FloatTok{0.5}\NormalTok{,}\FloatTok{0.95}\NormalTok{))}
\NormalTok{ df_angle<-}\KeywordTok{data.frame}\NormalTok{(}\DataTypeTok{Trip=}\KeywordTok{c}\NormalTok{(}\DecValTok{1}\OperatorTok{:}\KeywordTok{dim}\NormalTok{(m1a}\OperatorTok{$}\NormalTok{AlphaAngle)[}\DecValTok{2}\NormalTok{]),}\DataTypeTok{Group=}\StringTok{"Heading Change"}\NormalTok{,}\DataTypeTok{Group2=}\StringTok{"Mean"}\NormalTok{,}
                      \DataTypeTok{LI=}\NormalTok{sample_eff[}\DecValTok{1}\NormalTok{,],}
                      \DataTypeTok{Median=}\NormalTok{sample_eff[}\DecValTok{2}\NormalTok{,],}
                      \DataTypeTok{HI=}\NormalTok{sample_eff[}\DecValTok{3}\NormalTok{,])}
                      
\NormalTok{ m1d<-rstan}\OperatorTok{::}\KeywordTok{extract}\NormalTok{(m1,}\DataTypeTok{pars=}\StringTok{"AlphaDist"}\NormalTok{)}
\NormalTok{ sample_eff<-}\KeywordTok{apply}\NormalTok{(m1d}\OperatorTok{$}\NormalTok{AlphaDist,}\DecValTok{2}\NormalTok{,quantile,}\DataTypeTok{probs=}\KeywordTok{c}\NormalTok{(}\FloatTok{0.05}\NormalTok{,}\FloatTok{0.5}\NormalTok{,}\FloatTok{0.95}\NormalTok{))}
\NormalTok{ df_dist<-}\KeywordTok{data.frame}\NormalTok{(}\DataTypeTok{Trip=}\KeywordTok{c}\NormalTok{(}\DecValTok{1}\OperatorTok{:}\KeywordTok{dim}\NormalTok{(m1d}\OperatorTok{$}\NormalTok{AlphaDist)[}\DecValTok{2}\NormalTok{]),}\DataTypeTok{Group=}\StringTok{"Step-Size"}\NormalTok{,}\DataTypeTok{Group2=}\StringTok{"Mean"}\NormalTok{,}
                      \DataTypeTok{LI=}\NormalTok{sample_eff[}\DecValTok{1}\NormalTok{,],}
                      \DataTypeTok{Median=}\NormalTok{sample_eff[}\DecValTok{2}\NormalTok{,],}
                      \DataTypeTok{HI=}\NormalTok{sample_eff[}\DecValTok{3}\NormalTok{,])}
                      
\NormalTok{ df_all1<-}\KeywordTok{rbind}\NormalTok{(df_angle,df_dist)}

\NormalTok{  m1a<-rstan}\OperatorTok{::}\KeywordTok{extract}\NormalTok{(m1,}\DataTypeTok{pars=}\StringTok{"DAngle"}\NormalTok{)}
\NormalTok{ sample_eff<-}\KeywordTok{apply}\NormalTok{(}\KeywordTok{exp}\NormalTok{(m1a}\OperatorTok{$}\NormalTok{DAngle),}\DecValTok{2}\NormalTok{,quantile,}\DataTypeTok{probs=}\KeywordTok{c}\NormalTok{(}\FloatTok{0.05}\NormalTok{,}\FloatTok{0.5}\NormalTok{,}\FloatTok{0.95}\NormalTok{))}
\NormalTok{ df_angle<-}\KeywordTok{data.frame}\NormalTok{(}\DataTypeTok{Trip=}\KeywordTok{c}\NormalTok{(}\DecValTok{1}\OperatorTok{:}\KeywordTok{dim}\NormalTok{(m1a}\OperatorTok{$}\NormalTok{DAngle)[}\DecValTok{2}\NormalTok{]),}\DataTypeTok{Group=}\StringTok{"Heading Change"}\NormalTok{,}\DataTypeTok{Group2=}\StringTok{"Dispersion"}\NormalTok{,}
                      \DataTypeTok{LI=}\NormalTok{sample_eff[}\DecValTok{1}\NormalTok{,],}
                      \DataTypeTok{Median=}\NormalTok{sample_eff[}\DecValTok{2}\NormalTok{,],}
                      \DataTypeTok{HI=}\NormalTok{sample_eff[}\DecValTok{3}\NormalTok{,])}
                      
\NormalTok{ m1d<-rstan}\OperatorTok{::}\KeywordTok{extract}\NormalTok{(m1,}\DataTypeTok{pars=}\StringTok{"SDDist"}\NormalTok{)}
\NormalTok{ sample_eff<-}\KeywordTok{apply}\NormalTok{(}\KeywordTok{exp}\NormalTok{(m1d}\OperatorTok{$}\NormalTok{SDDist),}\DecValTok{2}\NormalTok{,quantile,}\DataTypeTok{probs=}\KeywordTok{c}\NormalTok{(}\FloatTok{0.05}\NormalTok{,}\FloatTok{0.5}\NormalTok{,}\FloatTok{0.95}\NormalTok{))}
\NormalTok{ df_dist<-}\KeywordTok{data.frame}\NormalTok{(}\DataTypeTok{Trip=}\KeywordTok{c}\NormalTok{(}\DecValTok{1}\OperatorTok{:}\KeywordTok{dim}\NormalTok{(m1d}\OperatorTok{$}\NormalTok{SDDist)[}\DecValTok{2}\NormalTok{]),}\DataTypeTok{Group=}\StringTok{"Step-Size"}\NormalTok{,}\DataTypeTok{Group2=}\StringTok{"Dispersion"}\NormalTok{,}
                      \DataTypeTok{LI=}\NormalTok{sample_eff[}\DecValTok{1}\NormalTok{,],}
                      \DataTypeTok{Median=}\NormalTok{sample_eff[}\DecValTok{2}\NormalTok{,],}
                      \DataTypeTok{HI=}\NormalTok{sample_eff[}\DecValTok{3}\NormalTok{,])}
                      
\NormalTok{ df_all2<-}\KeywordTok{rbind}\NormalTok{(df_angle,df_dist)}

\NormalTok{ df_all<-}\KeywordTok{rbind}\NormalTok{(df_all1,df_all2)}

 \KeywordTok{ggplot}\NormalTok{(df_all,}\KeywordTok{aes}\NormalTok{(}\DataTypeTok{x=}\NormalTok{Trip,}\DataTypeTok{y=}\NormalTok{Median))}\OperatorTok{+}\KeywordTok{geom_point}\NormalTok{()}\OperatorTok{+}
\StringTok{ }\KeywordTok{geom_linerange}\NormalTok{(}\KeywordTok{aes}\NormalTok{(}\DataTypeTok{ymin=}\NormalTok{LI,}\DataTypeTok{ymax=}\NormalTok{HI))}\OperatorTok{+}\KeywordTok{facet_wrap}\NormalTok{(Group2}\OperatorTok{~}\NormalTok{Group,}\DataTypeTok{scales=}\StringTok{"free"}\NormalTok{)}\OperatorTok{+}
\StringTok{ }\KeywordTok{labs}\NormalTok{(}\DataTypeTok{y=}\StringTok{"Regression parameters"}\NormalTok{) }\OperatorTok{+}\StringTok{ }\KeywordTok{theme}\NormalTok{(}\DataTypeTok{strip.text.x =} \KeywordTok{element_text}\NormalTok{(}\DataTypeTok{size=}\DecValTok{14}\NormalTok{,}\DataTypeTok{face=}\StringTok{"bold"}\NormalTok{),}\DataTypeTok{axis.text=}\KeywordTok{element_text}\NormalTok{(}\DataTypeTok{size=}\DecValTok{12}\NormalTok{),}
        \DataTypeTok{axis.title=}\KeywordTok{element_text}\NormalTok{(}\DataTypeTok{size=}\DecValTok{14}\NormalTok{,}\DataTypeTok{face=}\StringTok{"bold"}\NormalTok{)) }\OperatorTok{+}\StringTok{ }
\StringTok{ }\KeywordTok{scale_x_continuous}\NormalTok{(}\DataTypeTok{breaks=} \KeywordTok{pretty_breaks}\NormalTok{())}
\NormalTok{\}}


\NormalTok{### grackleate.R}
\NormalTok{grackleate <-}\StringTok{ }\ControlFlowTok{function}\NormalTok{(}\DataTypeTok{AlphaDist=}\FloatTok{2.8}\NormalTok{,}\DataTypeTok{AlphaAngle=}\DecValTok{0}\NormalTok{,}\DataTypeTok{BetaDist=}\KeywordTok{rep}\NormalTok{(}\DecValTok{0}\NormalTok{,}\DecValTok{10}\NormalTok{),}\DataTypeTok{BetaAngle=}\KeywordTok{rep}\NormalTok{(}\DecValTok{0}\NormalTok{,}\DecValTok{10}\NormalTok{),}\DataTypeTok{SDDist=}\DecValTok{1}\NormalTok{,}\DataTypeTok{DAngle=}\DecValTok{2}\NormalTok{,}\DataTypeTok{Thresh=}\DecValTok{50}\NormalTok{, }\DataTypeTok{Lags=}\DecValTok{10}\NormalTok{, }\DataTypeTok{steps=}\DecValTok{150}\NormalTok{, }\DataTypeTok{seed=}\DecValTok{123456}\NormalTok{, }\DataTypeTok{Reps=}\DecValTok{30}\NormalTok{,}\DataTypeTok{N_Patches=}\DecValTok{13}\NormalTok{)\{}

\NormalTok{################################################################################# Set up Patches of Food                                                }
\NormalTok{X_Mushrooms<-}\KeywordTok{vector}\NormalTok{(}\StringTok{"list"}\NormalTok{,N_Patches)                                           }\CommentTok{# Location of prey}
\NormalTok{Y_Mushrooms<-}\KeywordTok{vector}\NormalTok{(}\StringTok{"list"}\NormalTok{,N_Patches)                                           }\CommentTok{# }

\KeywordTok{set.seed}\NormalTok{(seed)                                                                  }\CommentTok{# Reset Seed}

\NormalTok{N_PerPatch<-}\KeywordTok{rpois}\NormalTok{(N_Patches, }\DecValTok{200}\NormalTok{)                                               }\CommentTok{# Create some prey}
\ControlFlowTok{for}\NormalTok{(i }\ControlFlowTok{in} \DecValTok{1}\OperatorTok{:}\NormalTok{N_Patches)\{}
\NormalTok{X_Mushrooms[[i]]<-}\KeywordTok{rnorm}\NormalTok{(N_PerPatch[i], }\KeywordTok{runif}\NormalTok{(}\DecValTok{1}\NormalTok{,}\OperatorTok{-}\DecValTok{2500}\NormalTok{,}\DecValTok{2500}\NormalTok{), }\KeywordTok{rpois}\NormalTok{(}\DecValTok{1}\NormalTok{, }\DecValTok{350}\NormalTok{))}\OperatorTok{+}\DecValTok{100}
\NormalTok{Y_Mushrooms[[i]]<-}\KeywordTok{rnorm}\NormalTok{(N_PerPatch[i], }\KeywordTok{runif}\NormalTok{(}\DecValTok{1}\NormalTok{,}\OperatorTok{-}\DecValTok{2500}\NormalTok{,}\DecValTok{2500}\NormalTok{), }\KeywordTok{rpois}\NormalTok{(}\DecValTok{1}\NormalTok{, }\DecValTok{350}\NormalTok{))}\OperatorTok{+}\DecValTok{100}
\NormalTok{\}}

\NormalTok{X_Mushrooms_per_step<-}\KeywordTok{vector}\NormalTok{(}\StringTok{"list"}\NormalTok{,steps)                                      }\CommentTok{# Location of prey}
\NormalTok{Y_Mushrooms_per_step<-}\KeywordTok{vector}\NormalTok{(}\StringTok{"list"}\NormalTok{,steps)                                      }\CommentTok{#}

\ControlFlowTok{for}\NormalTok{(i }\ControlFlowTok{in} \DecValTok{1}\OperatorTok{:}\NormalTok{steps)\{}
\NormalTok{ X_Mushrooms_per_step[[i]]<-X_Mushrooms}
\NormalTok{ Y_Mushrooms_per_step[[i]]<-Y_Mushrooms}
\NormalTok{\}                                                                 }
                                                                                     
\NormalTok{################################################################################ Start Model}
\NormalTok{Store.X <-}\StringTok{ }\KeywordTok{matrix}\NormalTok{(}\OtherTok{NA}\NormalTok{,}\DataTypeTok{ncol=}\NormalTok{Reps,}\DataTypeTok{nrow=}\NormalTok{steps}\OperatorTok{-}\DecValTok{1}\NormalTok{)}
\NormalTok{Store.Y <-}\StringTok{ }\KeywordTok{matrix}\NormalTok{(}\OtherTok{NA}\NormalTok{,}\DataTypeTok{ncol=}\NormalTok{Reps,}\DataTypeTok{nrow=}\NormalTok{steps}\OperatorTok{-}\DecValTok{1}\NormalTok{)}

\ControlFlowTok{for}\NormalTok{(q }\ControlFlowTok{in} \DecValTok{1}\OperatorTok{:}\NormalTok{Reps)\{                                                               }\CommentTok{# Run simulation several times}
\KeywordTok{set.seed}\NormalTok{(seed)                                                                  }\CommentTok{# Reset seed}

\NormalTok{N_PerPatch<-}\KeywordTok{rpois}\NormalTok{(N_Patches, }\DecValTok{200}\NormalTok{)                                               }\CommentTok{# Choose number of items per patch}

\ControlFlowTok{for}\NormalTok{(i }\ControlFlowTok{in} \DecValTok{1}\OperatorTok{:}\NormalTok{N_Patches)\{}
\NormalTok{X_Mushrooms[[i]]<-}\KeywordTok{rnorm}\NormalTok{(N_PerPatch[i], }\KeywordTok{runif}\NormalTok{(}\DecValTok{1}\NormalTok{,}\OperatorTok{-}\DecValTok{2500}\NormalTok{,}\DecValTok{2500}\NormalTok{), }\KeywordTok{rpois}\NormalTok{(}\DecValTok{1}\NormalTok{, }\DecValTok{350}\NormalTok{))}\OperatorTok{+}\DecValTok{100}  \CommentTok{# Make some patches of prey}
\NormalTok{Y_Mushrooms[[i]]<-}\KeywordTok{rnorm}\NormalTok{(N_PerPatch[i], }\KeywordTok{runif}\NormalTok{(}\DecValTok{1}\NormalTok{,}\OperatorTok{-}\DecValTok{2500}\NormalTok{,}\DecValTok{2500}\NormalTok{), }\KeywordTok{rpois}\NormalTok{(}\DecValTok{1}\NormalTok{, }\DecValTok{350}\NormalTok{))}\OperatorTok{+}\DecValTok{100}  \CommentTok{#}
\NormalTok{\}}

\NormalTok{X_Mushrooms_per_step<-}\KeywordTok{vector}\NormalTok{(}\StringTok{"list"}\NormalTok{,steps)                                      }\CommentTok{# Location of prey}
\NormalTok{Y_Mushrooms_per_step<-}\KeywordTok{vector}\NormalTok{(}\StringTok{"list"}\NormalTok{,steps)                                      }\CommentTok{#}

\ControlFlowTok{for}\NormalTok{(i }\ControlFlowTok{in} \DecValTok{1}\OperatorTok{:}\NormalTok{steps)\{}
\NormalTok{ X_Mushrooms_per_step[[i]]<-X_Mushrooms}
\NormalTok{ Y_Mushrooms_per_step[[i]]<-Y_Mushrooms}
\NormalTok{\}}

\NormalTok{loc.x<-}\KeywordTok{c}\NormalTok{()                                                                      }\CommentTok{# Location of forager}
\NormalTok{loc.y<-}\KeywordTok{c}\NormalTok{()                                                                      }\CommentTok{#}

\NormalTok{speed<-}\KeywordTok{c}\NormalTok{()                                                                      }\CommentTok{# Speed or step size}
\NormalTok{heading<-}\KeywordTok{c}\NormalTok{()                                                                    }\CommentTok{# Absolute heading}
\NormalTok{d.heading<-}\KeywordTok{c}\NormalTok{()                                                                  }\CommentTok{# Heading change}

\NormalTok{Hits<-}\KeywordTok{c}\NormalTok{()                                                                       }\CommentTok{# Binary vector of encounters}

\NormalTok{loc.x[}\DecValTok{1}\OperatorTok{:}\NormalTok{Lags]<-}\KeywordTok{rep}\NormalTok{(}\DecValTok{0}\NormalTok{,Lags)                                                      }\CommentTok{# Intialize vectors}
\NormalTok{loc.y[}\DecValTok{1}\OperatorTok{:}\NormalTok{Lags]<-}\KeywordTok{rep}\NormalTok{(}\DecValTok{0}\NormalTok{,Lags)                                                      }\CommentTok{#}
\NormalTok{heading[}\DecValTok{1}\OperatorTok{:}\NormalTok{Lags]<-}\KeywordTok{rep}\NormalTok{(}\DecValTok{0}\NormalTok{,Lags)                                                    }\CommentTok{#}
\NormalTok{speed[}\DecValTok{1}\OperatorTok{:}\NormalTok{Lags]<-}\KeywordTok{rep}\NormalTok{(}\DecValTok{0}\NormalTok{,Lags)                                                      }\CommentTok{#}
\NormalTok{Hits[}\DecValTok{1}\OperatorTok{:}\NormalTok{Lags]<-}\KeywordTok{rep}\NormalTok{(}\DecValTok{0}\NormalTok{,Lags)                                                       }\CommentTok{#}

\KeywordTok{plot}\NormalTok{(loc.x,loc.y,}\DataTypeTok{typ=}\StringTok{"l"}\NormalTok{,}\DataTypeTok{ylim=}\KeywordTok{c}\NormalTok{(}\OperatorTok{-}\DecValTok{3500}\NormalTok{,}\DecValTok{3500}\NormalTok{),}\DataTypeTok{xlim=}\KeywordTok{c}\NormalTok{(}\OperatorTok{-}\DecValTok{3500}\NormalTok{,}\DecValTok{3500}\NormalTok{))                 }\CommentTok{# Plot prey items}
 \ControlFlowTok{for}\NormalTok{(i }\ControlFlowTok{in} \DecValTok{1}\OperatorTok{:}\NormalTok{N_Patches)\{                                                         }\CommentTok{#}
  \KeywordTok{points}\NormalTok{(X_Mushrooms[[i]],Y_Mushrooms[[i]], }\DataTypeTok{col=}\StringTok{"red"}\NormalTok{,}\DataTypeTok{pch=}\StringTok{"."}\NormalTok{)                  }\CommentTok{#}
\NormalTok{   \}                                                                            }\CommentTok{#}

\KeywordTok{set.seed}\NormalTok{(q}\OperatorTok{*}\DecValTok{103}\NormalTok{)}

\NormalTok{################################################################################ Now model forager movement}

\ControlFlowTok{for}\NormalTok{(s }\ControlFlowTok{in}\NormalTok{ (Lags}\OperatorTok{+}\DecValTok{1}\NormalTok{)}\OperatorTok{:}\StringTok{ }\NormalTok{(steps}\OperatorTok{-}\DecValTok{1}\NormalTok{))\{                                                  }
\NormalTok{ X_Mushrooms <-}\StringTok{ }\NormalTok{X_Mushrooms_per_step[[s]]}
\NormalTok{ Y_Mushrooms <-}\StringTok{ }\NormalTok{Y_Mushrooms_per_step[[s]]}

\NormalTok{ PredDist <-}\StringTok{ }\NormalTok{AlphaDist;                                                         }\CommentTok{# First calculate mean prediction conditional on encounters}
\ControlFlowTok{for}\NormalTok{(k }\ControlFlowTok{in} \DecValTok{1}\OperatorTok{:}\NormalTok{Lags)\{                                                               }\CommentTok{#    }
\NormalTok{ PredDist <-}\StringTok{ }\NormalTok{PredDist }\OperatorTok{+}\StringTok{ }\NormalTok{BetaDist[k]}\OperatorTok{*}\KeywordTok{ifelse}\NormalTok{(Hits[s}\OperatorTok{-}\NormalTok{k]}\OperatorTok{>}\DecValTok{0}\NormalTok{,}\DecValTok{1}\NormalTok{,}\DecValTok{0}\NormalTok{);                                  }\CommentTok{#}
\NormalTok{       \}                                                                        }\CommentTok{#}

\NormalTok{ R<-}\StringTok{ }\KeywordTok{exp}\NormalTok{(}\KeywordTok{rnorm}\NormalTok{(}\DecValTok{1}\NormalTok{,PredDist,SDDist))                                              }\CommentTok{# Then simulate a step distance. }
    
\NormalTok{ PredAngle <-}\StringTok{ }\NormalTok{AlphaAngle;                                                       }\CommentTok{# Again calculate mean prediction conditional on encounters}
\ControlFlowTok{for}\NormalTok{(k }\ControlFlowTok{in} \DecValTok{1}\OperatorTok{:}\NormalTok{Lags)\{                                                               }\CommentTok{#    }
\NormalTok{ PredAngle <-}\StringTok{ }\NormalTok{PredAngle }\OperatorTok{+}\StringTok{ }\NormalTok{BetaAngle[k]}\OperatorTok{*}\KeywordTok{ifelse}\NormalTok{(Hits[s}\OperatorTok{-}\NormalTok{k]}\OperatorTok{>}\DecValTok{0}\NormalTok{,}\DecValTok{1}\NormalTok{,}\DecValTok{0}\NormalTok{);                              }\CommentTok{#}
\NormalTok{        \}                                                                       }\CommentTok{#}
                
\NormalTok{ Theta<-}\StringTok{ }\KeywordTok{rbeta}\NormalTok{(}\DecValTok{1}\NormalTok{,}\KeywordTok{inv_logit}\NormalTok{(PredAngle)}\OperatorTok{*}\NormalTok{DAngle,                                   }\CommentTok{# And then simulate a directional change}
\NormalTok{          (}\DecValTok{1}\OperatorTok{-}\KeywordTok{inv_logit}\NormalTok{(PredAngle))}\OperatorTok{*}\NormalTok{DAngle)}\OperatorTok{*}\DecValTok{180}\OperatorTok{*}\KeywordTok{ifelse}\NormalTok{(}\KeywordTok{runif}\NormalTok{(}\DecValTok{1}\NormalTok{,}\DecValTok{0}\NormalTok{,}\DecValTok{1}\NormalTok{)}\OperatorTok{>}\FloatTok{0.5}\NormalTok{,}\DecValTok{1}\NormalTok{,}\OperatorTok{-}\DecValTok{1}\NormalTok{)    }\CommentTok{# The 180 shifts from the unit to the maximum absolute distance}
                                                                                \CommentTok{# The ifelse just chooses a random direction}
                                                                             
\NormalTok{ heading[s]<-(heading[s}\OperatorTok{-}\DecValTok{1}\NormalTok{]}\OperatorTok{+}\NormalTok{Theta)}\OperatorTok\DecValTok{360}                                          \CommentTok{# Store new heading. Note that the %% is the mod operation to wrap around if needed}
\NormalTok{ d.heading[s] <-}\StringTok{ }\KeywordTok{abs}\NormalTok{(Theta)}\OperatorTok{/}\DecValTok{180}                                                 \CommentTok{# Also store just the delta heading}
\NormalTok{ speed[s] <-}\StringTok{ }\NormalTok{R                                                                  }\CommentTok{# And the speed slash step size}
   
\NormalTok{ ynew <-}\StringTok{ }\NormalTok{R }\OperatorTok{*}\StringTok{ }\KeywordTok{sin}\NormalTok{(}\KeywordTok{deg2rad}\NormalTok{(heading[s]))                                           }\CommentTok{# Now convert polar to Cartesian, to get the offset for a new x and y pair}
\NormalTok{ xnew <-}\StringTok{ }\NormalTok{R }\OperatorTok{*}\StringTok{ }\KeywordTok{cos}\NormalTok{(}\KeywordTok{deg2rad}\NormalTok{(heading[s]))                                           }\CommentTok{#}
   
\NormalTok{ loc.x[s]<-loc.x[s}\OperatorTok{-}\DecValTok{1}\NormalTok{]}\OperatorTok{+}\NormalTok{xnew                                                      }\CommentTok{# Make the new x and y pair}
\NormalTok{ loc.y[s]<-loc.y[s}\OperatorTok{-}\DecValTok{1}\NormalTok{]}\OperatorTok{+}\NormalTok{ynew                                                      }\CommentTok{#}

\NormalTok{############################################################################### Now check for an encounter }
\NormalTok{ Scrap2<-}\KeywordTok{c}\NormalTok{()}
 \ControlFlowTok{for}\NormalTok{(i }\ControlFlowTok{in} \DecValTok{1}\OperatorTok{:}\NormalTok{N_Patches)\{                                                         }\CommentTok{# For each patch}
\NormalTok{  Scrap<-}\KeywordTok{rep}\NormalTok{(}\OtherTok{NA}\NormalTok{,}\KeywordTok{length}\NormalTok{(X_Mushrooms[[i]]))}
 \ControlFlowTok{for}\NormalTok{(j }\ControlFlowTok{in} \DecValTok{1}\OperatorTok{:}\KeywordTok{length}\NormalTok{(X_Mushrooms[[i]]))\{                                          }\CommentTok{# For each mushroom in patch}
\NormalTok{ Scrap[j]<-}\StringTok{ }\KeywordTok{dist}\NormalTok{(loc.x[s],X_Mushrooms[[i]][j],loc.y[s],Y_Mushrooms[[i]][j]);    }\CommentTok{# Calculate the distance from the forager to the mushroom}
 \ControlFlowTok{if}\NormalTok{(Scrap[j]}\OperatorTok{<}\NormalTok{Thresh)\{                                                           }\CommentTok{# If the forager is closer than the visual radius slash encounter threshold}
 \KeywordTok{points}\NormalTok{(X_Mushrooms[[i]][j],Y_Mushrooms[[i]][j], }\DataTypeTok{col=}\StringTok{"blue"}\NormalTok{,}\DataTypeTok{pch=}\DecValTok{20}\NormalTok{)             }\CommentTok{# Plot a hit}

       \ControlFlowTok{for}\NormalTok{(allgone }\ControlFlowTok{in}\NormalTok{ s}\OperatorTok{:}\StringTok{ }\NormalTok{(steps}\OperatorTok{-}\DecValTok{1}\NormalTok{))\{}
\NormalTok{       X_Mushrooms_per_step[[allgone]][[i]][j]<-}\DecValTok{99999}                           \CommentTok{# If this run is for destructive foraging}
\NormalTok{       X_Mushrooms_per_step[[allgone]][[i]][j]<-}\DecValTok{99999}                           \CommentTok{# disappear food for all future timesteps}
\NormalTok{      \}}
\NormalTok{ \}\}}
 
\NormalTok{ Scrap2[i]<-}\StringTok{ }\KeywordTok{ifelse}\NormalTok{(}\KeywordTok{sum}\NormalTok{(}\KeywordTok{ifelse}\NormalTok{(Scrap}\OperatorTok{<}\NormalTok{Thresh,}\DecValTok{1}\NormalTok{,}\DecValTok{0}\NormalTok{))}\OperatorTok{>}\DecValTok{0}\NormalTok{,}\KeywordTok{sum}\NormalTok{(}\KeywordTok{ifelse}\NormalTok{(Scrap}\OperatorTok{<}\NormalTok{Thresh,}\DecValTok{1}\NormalTok{,}\DecValTok{0}\NormalTok{)),}\DecValTok{0}\NormalTok{)      }\CommentTok{# Check for hits, also can replace sum(ifelse(Scrap<Thresh,1,0)) with 1}
\NormalTok{ \}}
 
\NormalTok{ Hits[s]<-}\KeywordTok{ifelse}\NormalTok{(}\KeywordTok{sum}\NormalTok{(Scrap2)}\OperatorTok{==}\DecValTok{0}\NormalTok{,}\DecValTok{0}\NormalTok{,}\KeywordTok{sum}\NormalTok{(Scrap2))                                            }\CommentTok{# If there is a hit, then set encounters to 1}
  
 \KeywordTok{lines}\NormalTok{(loc.x,loc.y,}\DataTypeTok{ylim=}\KeywordTok{c}\NormalTok{(}\OperatorTok{-}\DecValTok{2500}\NormalTok{,}\DecValTok{2500}\NormalTok{),}\DataTypeTok{xlim=}\KeywordTok{c}\NormalTok{(}\OperatorTok{-}\DecValTok{2500}\NormalTok{,}\DecValTok{2500}\NormalTok{))                                 }\CommentTok{# Plot updates to the foragers path}
\NormalTok{\}}
\NormalTok{ Store.X[,q] <-}\StringTok{ }\NormalTok{loc.x}
\NormalTok{ Store.Y[,q] <-}\StringTok{ }\NormalTok{loc.y}
\NormalTok{ \}}


 \KeywordTok{return}\NormalTok{(}\KeywordTok{list}\NormalTok{(}\DataTypeTok{X=}\NormalTok{Store.X[(Lags}\OperatorTok{+}\DecValTok{1}\NormalTok{)}\OperatorTok{:}\KeywordTok{length}\NormalTok{(loc.x),],}\DataTypeTok{Y=}\NormalTok{Store.Y[(Lags}\OperatorTok{+}\DecValTok{1}\NormalTok{)}\OperatorTok{:}\KeywordTok{length}\NormalTok{(loc.y),]))}
\NormalTok{ \}}


\NormalTok{### grackleations.R}
\NormalTok{grackleations <-}\StringTok{ }\ControlFlowTok{function}\NormalTok{(m2)\{}
\NormalTok{df2<-}\KeywordTok{data.frame}\NormalTok{(}\DataTypeTok{Parameter=}\StringTok{"B"}\NormalTok{,}\DataTypeTok{Value=}\NormalTok{rstan}\OperatorTok{::}\KeywordTok{extract}\NormalTok{(m2,}\DataTypeTok{pars=}\StringTok{"B"}\NormalTok{)}\OperatorTok{$}\NormalTok{B) }\CommentTok{# Mean Step-Size over Days}


 \KeywordTok{ggplot}\NormalTok{(df2, }\KeywordTok{aes}\NormalTok{(}\DataTypeTok{x=}\NormalTok{Value)) }\OperatorTok{+}\StringTok{ }\KeywordTok{geom_density}\NormalTok{(}\KeywordTok{aes}\NormalTok{(}\DataTypeTok{fill=}\NormalTok{Parameter), }\DataTypeTok{alpha=}\FloatTok{0.3}\NormalTok{) }\OperatorTok{+}
\StringTok{ }\KeywordTok{theme}\NormalTok{(}\DataTypeTok{strip.text.x =} \KeywordTok{element_text}\NormalTok{(}\DataTypeTok{size=}\DecValTok{14}\NormalTok{,}\DataTypeTok{face=}\StringTok{"bold"}\NormalTok{),}\DataTypeTok{axis.text=}\KeywordTok{element_text}\NormalTok{(}\DataTypeTok{size=}\DecValTok{12}\NormalTok{),}
        \DataTypeTok{axis.title=}\KeywordTok{element_text}\NormalTok{(}\DataTypeTok{size=}\DecValTok{14}\NormalTok{,}\DataTypeTok{face=}\StringTok{"bold"}\NormalTok{)) }
\NormalTok{\}}


\NormalTok{### gracklebinner.R}
\NormalTok{gracklebinner <-}\StringTok{ }\ControlFlowTok{function}\NormalTok{(tracks,}\DataTypeTok{nbin=}\KeywordTok{c}\NormalTok{(}\DecValTok{15}\NormalTok{,}\DecValTok{15}\NormalTok{),}\DataTypeTok{ab_override=}\OtherTok{NULL}\NormalTok{)\{}

\NormalTok{Trips <-}\StringTok{ }\KeywordTok{dim}\NormalTok{(tracks}\OperatorTok{$}\NormalTok{X)[}\DecValTok{2}\NormalTok{]}
\NormalTok{GrackBins <-}\StringTok{ }\KeywordTok{matrix}\NormalTok{(}\OtherTok{NA}\NormalTok{,}\DataTypeTok{nrow=}\NormalTok{Trips,}\DataTypeTok{ncol=}\NormalTok{nbin[}\DecValTok{1}\NormalTok{]}\OperatorTok{*}\NormalTok{nbin[}\DecValTok{2}\NormalTok{])}
\NormalTok{bins<-}\KeywordTok{bin2}\NormalTok{(}\KeywordTok{cbind}\NormalTok{(}\KeywordTok{c}\NormalTok{(z}\OperatorTok{$}\NormalTok{X),}\KeywordTok{c}\NormalTok{(z}\OperatorTok{$}\NormalTok{Y)),}\DataTypeTok{nbin=}\NormalTok{nbin)}

\ControlFlowTok{if}\NormalTok{(}\KeywordTok{length}\NormalTok{(}\KeywordTok{dim}\NormalTok{(ab_override))}\OperatorTok{==}\DecValTok{0}\NormalTok{)\{}
\ControlFlowTok{for}\NormalTok{( i }\ControlFlowTok{in} \DecValTok{1}\OperatorTok{:}\NormalTok{Trips)\{}
\NormalTok{GrackBins[i,] <-}\StringTok{ }\KeywordTok{c}\NormalTok{(}\KeywordTok{bin2}\NormalTok{(}\KeywordTok{cbind}\NormalTok{(z}\OperatorTok{$}\NormalTok{X[,i],z}\OperatorTok{$}\NormalTok{Y[,i]),}\DataTypeTok{nbin=}\NormalTok{nbin,}\DataTypeTok{ab=}\NormalTok{bins}\OperatorTok{$}\NormalTok{ab)}\OperatorTok{$}\NormalTok{nc)}
\NormalTok{ \}}
\NormalTok{ \} }\ControlFlowTok{else}\NormalTok{\{}
\ControlFlowTok{for}\NormalTok{( i }\ControlFlowTok{in} \DecValTok{1}\OperatorTok{:}\NormalTok{Trips)\{}
\NormalTok{GrackBins[i,] <-}\StringTok{ }\KeywordTok{c}\NormalTok{(}\KeywordTok{bin2}\NormalTok{(}\KeywordTok{cbind}\NormalTok{(z}\OperatorTok{$}\NormalTok{X[,i],z}\OperatorTok{$}\NormalTok{Y[,i]),}\DataTypeTok{nbin=}\NormalTok{nbin,}\DataTypeTok{ab=}\NormalTok{ab_override)}\OperatorTok{$}\NormalTok{nc)}
\NormalTok{ \}}
\NormalTok{ \}}

\KeywordTok{return}\NormalTok{(GrackBins)}
\NormalTok{\}}


\NormalTok{### grackleize.R}
\NormalTok{grackleize <-}\StringTok{ }\ControlFlowTok{function}\NormalTok{(StoreX,StoreY) \{                                          }
\NormalTok{############################################################ Prepare Data}
\NormalTok{ Trip <-}\StringTok{ }\KeywordTok{dim}\NormalTok{(StoreX)[}\DecValTok{2}\NormalTok{]  }
\NormalTok{ MaxTicks<-}\KeywordTok{dim}\NormalTok{(StoreX)[}\DecValTok{1}\NormalTok{]}
    
\NormalTok{ Distance <-}\KeywordTok{matrix}\NormalTok{(}\OtherTok{NA}\NormalTok{,}\DataTypeTok{ncol=}\KeywordTok{max}\NormalTok{(Trip),}\DataTypeTok{nrow=}\NormalTok{MaxTicks) }
\NormalTok{ Ang <-}\KeywordTok{matrix}\NormalTok{(}\OtherTok{NA}\NormalTok{,}\DataTypeTok{ncol=}\KeywordTok{max}\NormalTok{(Trip),}\DataTypeTok{nrow=}\NormalTok{MaxTicks)}
\NormalTok{ AngDiff <-}\KeywordTok{matrix}\NormalTok{(}\OtherTok{NA}\NormalTok{,}\DataTypeTok{ncol=}\KeywordTok{max}\NormalTok{(Trip),}\DataTypeTok{nrow=}\NormalTok{MaxTicks)}

\NormalTok{ N<-}\KeywordTok{c}\NormalTok{()}
 \ControlFlowTok{for}\NormalTok{(j }\ControlFlowTok{in} \DecValTok{1}\OperatorTok{:}\KeywordTok{max}\NormalTok{(Trip))\{}
\NormalTok{ X <-}\StringTok{ }\NormalTok{StoreX[,j]}
\NormalTok{ Y <-}\StringTok{ }\NormalTok{StoreY[,j]}
\NormalTok{ N[j] <-}\StringTok{ }\KeywordTok{length}\NormalTok{(X)}
  
\NormalTok{ Dist<-}\KeywordTok{c}\NormalTok{()  }
\NormalTok{ ang1<-}\KeywordTok{c}\NormalTok{()}
\NormalTok{ angdif<-}\KeywordTok{c}\NormalTok{()}
  
 \ControlFlowTok{for}\NormalTok{(i }\ControlFlowTok{in} \DecValTok{2}\OperatorTok{:}\NormalTok{N[j])\{}
\NormalTok{ Dist[i]<-}\StringTok{ }\KeywordTok{dist}\NormalTok{(X[i],X[i}\OperatorTok{-}\DecValTok{1}\NormalTok{],Y[i],Y[i}\OperatorTok{-}\DecValTok{1}\NormalTok{]);}
\NormalTok{ ang1[i]<-}\StringTok{ }\KeywordTok{ang}\NormalTok{(X[i],X[i}\OperatorTok{-}\DecValTok{1}\NormalTok{],Y[i],Y[i}\OperatorTok{-}\DecValTok{1}\NormalTok{])}
 \ControlFlowTok{if}\NormalTok{(i}\OperatorTok{>}\DecValTok{3}\NormalTok{)\{}
\NormalTok{ angdif[i]<-}\StringTok{ }\KeywordTok{ang.dif}\NormalTok{(ang1[i],ang1[i}\OperatorTok{-}\DecValTok{1}\NormalTok{])}
\NormalTok{         \}\}    }
         
\NormalTok{ Distance[}\DecValTok{1}\OperatorTok{:}\NormalTok{N[j],j]<-Dist}
\NormalTok{ Ang[}\DecValTok{1}\OperatorTok{:}\NormalTok{N[j],j]<-ang1       }
\NormalTok{ AngDiff[}\DecValTok{1}\OperatorTok{:}\NormalTok{N[j],j]<-angdif}
\NormalTok{ \}}

 \ControlFlowTok{for}\NormalTok{(i }\ControlFlowTok{in} \DecValTok{1}\OperatorTok{:}\NormalTok{MaxTicks)\{}
 \ControlFlowTok{for}\NormalTok{(j }\ControlFlowTok{in} \DecValTok{1}\OperatorTok{:}\KeywordTok{max}\NormalTok{(Trip))\{}
\NormalTok{  Distance[i,j] <-}\StringTok{ }\KeywordTok{ifelse}\NormalTok{(Distance[i,j]}\OperatorTok{==}\DecValTok{0}\NormalTok{,}
                          \KeywordTok{runif}\NormalTok{(}\DecValTok{1}\NormalTok{,}\DecValTok{0}\NormalTok{,}\KeywordTok{min}\NormalTok{(Distance[}\KeywordTok{which}\NormalTok{(Distance }\OperatorTok{!=}\StringTok{ }\DecValTok{0}\NormalTok{)],}\DataTypeTok{na.rm=}\OtherTok{TRUE}\NormalTok{)),}
\NormalTok{                          Distance[i,j])}
\NormalTok{      \}\}}
      

\NormalTok{ Distance[}\KeywordTok{is.na}\NormalTok{(Distance)] <-}\StringTok{ }\DecValTok{999999}

\NormalTok{ EmptyAngle <-}\StringTok{  }\KeywordTok{fitdistr}\NormalTok{( }\KeywordTok{ScaleBeta}\NormalTok{(}\KeywordTok{c}\NormalTok{(}\KeywordTok{na.omit}\NormalTok{(}\KeywordTok{c}\NormalTok{(AngDiff)))),}\DataTypeTok{start=}\KeywordTok{list}\NormalTok{(}\DataTypeTok{shape1=}\DecValTok{1}\NormalTok{,}\DataTypeTok{shape2=}\DecValTok{1}\NormalTok{),}\StringTok{"beta"}\NormalTok{)}\OperatorTok{$}\NormalTok{estimate}

 \ControlFlowTok{for}\NormalTok{(i }\ControlFlowTok{in} \DecValTok{1}\OperatorTok{:}\NormalTok{MaxTicks)\{}
 \ControlFlowTok{for}\NormalTok{(j }\ControlFlowTok{in} \DecValTok{1}\OperatorTok{:}\KeywordTok{max}\NormalTok{(Trip))\{}
\NormalTok{  AngDiff[i,j] <-}\StringTok{ }\KeywordTok{ifelse}\NormalTok{(}\KeywordTok{is.nan}\NormalTok{(AngDiff[i,j]), }\KeywordTok{rbeta}\NormalTok{(}\DecValTok{1}\NormalTok{,EmptyAngle[}\DecValTok{1}\NormalTok{], EmptyAngle[}\DecValTok{2}\NormalTok{])}\OperatorTok{*}\NormalTok{pi,AngDiff[i,j])}
\NormalTok{      \}\}}
      
\NormalTok{ AngDiff <-}\StringTok{ }\KeywordTok{ScaleBeta}\NormalTok{(AngDiff)}
\NormalTok{ AngDiff[}\KeywordTok{is.na}\NormalTok{(AngDiff)] <-}\StringTok{ }\DecValTok{999999}

\NormalTok{ AngDiff <-}\StringTok{ }\NormalTok{AngDiff[}\KeywordTok{c}\NormalTok{(}\OperatorTok{-}\DecValTok{1}\NormalTok{,}\OperatorTok{-}\DecValTok{2}\NormalTok{,}\OperatorTok{-}\DecValTok{3}\NormalTok{),]}
\NormalTok{ Distance <-}\StringTok{ }\NormalTok{Distance[}\KeywordTok{c}\NormalTok{(}\OperatorTok{-}\DecValTok{1}\NormalTok{,}\OperatorTok{-}\DecValTok{2}\NormalTok{,}\OperatorTok{-}\DecValTok{3}\NormalTok{),]}
\NormalTok{ N <-}\StringTok{ }\NormalTok{N}\OperatorTok{-}\DecValTok{3}
\NormalTok{ MaxTicks<-MaxTicks}\OperatorTok{-}\DecValTok{3}


\NormalTok{########################################################## Extract data for Stan    }
\NormalTok{model_dat=}\KeywordTok{list}\NormalTok{(}
\DataTypeTok{Distance=}\NormalTok{Distance, }
\DataTypeTok{AngDiff=}\NormalTok{AngDiff,}
\DataTypeTok{N=}\NormalTok{N,}
\DataTypeTok{MaxTrip=}\KeywordTok{max}\NormalTok{(Trip),}
\DataTypeTok{MaxTicks=}\NormalTok{MaxTicks)}

\NormalTok{################################################################# Fit Stan Model}
\NormalTok{model_code_}\DecValTok{1}\NormalTok{ <-}\StringTok{ '}
\StringTok{data\{}
\StringTok{int MaxTrip;}
\StringTok{int MaxTicks;}
\StringTok{int N[MaxTrip];}

\StringTok{real Distance[MaxTicks,MaxTrip];}
\StringTok{real AngDiff[MaxTicks,MaxTrip];}
\StringTok{\}}

\StringTok{parameters \{}
\StringTok{ real MuD_M;}
\StringTok{ real MuA_M;}

\StringTok{ real<lower=0> DispD_M;}
\StringTok{ real<lower=0> DispA_M;}

\StringTok{ real MuD_D;}
\StringTok{ real MuA_D;}

\StringTok{ real<lower=0> DispD_D;}
\StringTok{ real<lower=0> DispA_D;}

\StringTok{ vector[MaxTrip] AlphaDist_Raw;}
\StringTok{ vector[MaxTrip] SDDist_Raw; }

\StringTok{ vector[MaxTrip] AlphaAngle_Raw;}
\StringTok{ vector[MaxTrip] DAngle_Raw; }
\StringTok{\}}

\StringTok{transformed parameters\{}
\StringTok{ vector[MaxTrip] AlphaDist;}
\StringTok{ vector[MaxTrip] SDDist; }

\StringTok{ vector[MaxTrip] AlphaAngle;}
\StringTok{ vector[MaxTrip] DAngle; }

\StringTok{ AlphaDist = MuD_M + DispD_M*AlphaDist_Raw;}
\StringTok{ SDDist = MuD_D + DispD_D*SDDist_Raw;}

\StringTok{ AlphaAngle = MuA_M + DispA_M*AlphaAngle_Raw;}
\StringTok{ DAngle = MuA_D + DispA_D*DAngle_Raw;  }

\StringTok{\}}

\StringTok{model\{}
\StringTok{  MuD_M~normal(0,5); }
\StringTok{  MuA_M~normal(0,5); }

\StringTok{  DispD_M~cauchy(0,1); }
\StringTok{  DispA_M~cauchy(0,1); }

\StringTok{  MuD_D~normal(0,5); }
\StringTok{  MuA_D~normal(0,5); }

\StringTok{  DispD_D~cauchy(0,1); }
\StringTok{  DispA_D~cauchy(0,1); }

\StringTok{  AlphaDist_Raw~normal(0,1); }
\StringTok{  SDDist_Raw~normal(0,1); }

\StringTok{  AlphaAngle_Raw~normal(0,1); }
\StringTok{  DAngle_Raw~normal(0,1); }

\StringTok{for(j in 1:MaxTrip)\{}
\StringTok{\{}

\StringTok{ vector[N[j]] PredAngle;}
\StringTok{ vector[N[j]] PredDist;}
\StringTok{ vector[N[j]] Dist;}
\StringTok{ vector[N[j]] AngleDifference;}
\StringTok{  }
\StringTok{ for(i in 1:N[j])\{}
\StringTok{  PredDist[i] = AlphaDist[j]; }
\StringTok{  \} }
\StringTok{        }
\StringTok{ for(i in 1:N[j])\{}
\StringTok{  PredAngle[i] = AlphaAngle[j]; }
\StringTok{  \}       }
\StringTok{              }
\StringTok{ for(i in 1:N[j])\{            }
\StringTok{  Dist[i] = Distance[i,j];              }
\StringTok{  AngleDifference[i] = AngDiff[i,j];}
\StringTok{  \}}

\StringTok{ Dist ~ lognormal(PredDist,exp(SDDist[j]));}
\StringTok{ AngleDifference ~ beta(inv_logit(PredAngle)*exp(DAngle[j]), (1-inv_logit(PredAngle))*exp(DAngle[j]));}
\StringTok{ \}\}    }

\StringTok{\}}

\StringTok{'}

\NormalTok{ m1 <-}\StringTok{ }\KeywordTok{stan}\NormalTok{( }\DataTypeTok{model_code=}\NormalTok{model_code_}\DecValTok{1}\NormalTok{, }\DataTypeTok{data=}\NormalTok{model_dat,}\DataTypeTok{refresh=}\DecValTok{1}\NormalTok{,}\DataTypeTok{chains=}\DecValTok{1}\NormalTok{, }\DataTypeTok{control =} \KeywordTok{list}\NormalTok{(}\DataTypeTok{adapt_delta =} \FloatTok{0.9}\NormalTok{, }\DataTypeTok{max_treedepth =} \DecValTok{15}\NormalTok{))}

 \KeywordTok{return}\NormalTok{(m1)}
\NormalTok{\}}



\NormalTok{### setup.R}
\CommentTok{# Load libraries}
     \KeywordTok{library}\NormalTok{(MASS)}
     \KeywordTok{library}\NormalTok{(mvtnorm)}
     \KeywordTok{library}\NormalTok{(fields)}
     \KeywordTok{library}\NormalTok{(ggplot2)}
     \KeywordTok{library}\NormalTok{(rethinking)  }
     \KeywordTok{library}\NormalTok{(sfsmisc)}
     \KeywordTok{library}\NormalTok{(ash)}
     \KeywordTok{library}\NormalTok{(reshape)}
     \KeywordTok{library}\NormalTok{(maptools)}
     \KeywordTok{library}\NormalTok{(gridExtra)}
     \KeywordTok{library}\NormalTok{(scales)}
     \KeywordTok{library}\NormalTok{(msm)}
     \KeywordTok{library}\NormalTok{(maps)}
     \KeywordTok{library}\NormalTok{(grid)}
     \KeywordTok{library}\NormalTok{(xtable)}
    
    \CommentTok{# Define functions}
\NormalTok{     dist2<-}\ControlFlowTok{function}\NormalTok{(a,b)\{}\KeywordTok{return}\NormalTok{(}\KeywordTok{sqrt}\NormalTok{( (b[}\DecValTok{2}\NormalTok{]}\OperatorTok{-}\NormalTok{a[}\DecValTok{2}\NormalTok{])}\OperatorTok{^}\DecValTok{2}   \OperatorTok{+}\StringTok{ }\NormalTok{(b[}\DecValTok{1}\NormalTok{]}\OperatorTok{-}\NormalTok{a[}\DecValTok{1}\NormalTok{])}\OperatorTok{^}\DecValTok{2}\NormalTok{ ))\}  }
    
\NormalTok{     dist<-}\ControlFlowTok{function}\NormalTok{(x2a,x1a,y2a,y1a)\{}\KeywordTok{return}\NormalTok{(}\KeywordTok{sqrt}\NormalTok{((x2a}\OperatorTok{-}\NormalTok{x1a)}\OperatorTok{^}\DecValTok{2} \OperatorTok{+}\StringTok{ }\NormalTok{(y2a}\OperatorTok{-}\NormalTok{y1a)}\OperatorTok{^}\DecValTok{2}\NormalTok{))\}  }
    
\NormalTok{     rad2deg <-}\StringTok{ }\ControlFlowTok{function}\NormalTok{(rad) \{(rad }\OperatorTok{*}\StringTok{ }\DecValTok{180}\NormalTok{) }\OperatorTok{/}\StringTok{ }\NormalTok{(pi)\}}
    
\NormalTok{     deg2rad <-}\StringTok{ }\ControlFlowTok{function}\NormalTok{(deg) \{(deg }\OperatorTok{*}\StringTok{ }\NormalTok{pi) }\OperatorTok{/}\StringTok{ }\NormalTok{(}\DecValTok{180}\NormalTok{)\}}
    
\NormalTok{     ang.dif <-}\StringTok{ }\ControlFlowTok{function}\NormalTok{(x,y) \{}\KeywordTok{min}\NormalTok{((}\DecValTok{2} \OperatorTok{*}\StringTok{ }\NormalTok{pi) }\OperatorTok{-}\StringTok{ }\KeywordTok{abs}\NormalTok{(x }\OperatorTok{-}\StringTok{ }\NormalTok{y), }\KeywordTok{abs}\NormalTok{(x }\OperatorTok{-}\StringTok{ }\NormalTok{y))\}}
    
\NormalTok{     lp_dist <-}\ControlFlowTok{function}\NormalTok{(a,b,c)\{ }
                               \CommentTok{# Return distance between a point and line segment }
\NormalTok{                               t <-}\StringTok{ }\OperatorTok{-}\NormalTok{(((a[}\DecValTok{1}\NormalTok{]}\OperatorTok{-}\NormalTok{c[}\DecValTok{1}\NormalTok{])}\OperatorTok{*}\NormalTok{(b[}\DecValTok{1}\NormalTok{]}\OperatorTok{-}\NormalTok{a[}\DecValTok{1}\NormalTok{]) }\OperatorTok{+}\StringTok{ }\NormalTok{(a[}\DecValTok{2}\NormalTok{]}\OperatorTok{-}\NormalTok{c[}\DecValTok{2}\NormalTok{])}\OperatorTok{*}\NormalTok{(b[}\DecValTok{2}\NormalTok{]}\OperatorTok{-}\NormalTok{a[}\DecValTok{2}\NormalTok{])) }\OperatorTok{/}\StringTok{ }\NormalTok{((b[}\DecValTok{1}\NormalTok{]}\OperatorTok{-}\NormalTok{a[}\DecValTok{1}\NormalTok{])}\OperatorTok{^}\DecValTok{2} \OperatorTok{+}\StringTok{ }\NormalTok{(b[}\DecValTok{2}\NormalTok{] }\OperatorTok{-}\StringTok{ }\NormalTok{a[}\DecValTok{2}\NormalTok{])}\OperatorTok{^}\DecValTok{2}\NormalTok{ ))}
    
                               \ControlFlowTok{if}\NormalTok{(t }\OperatorTok{>}\DecValTok{0} \OperatorTok{&}\StringTok{ }\NormalTok{t }\OperatorTok{<}\DecValTok{1}\NormalTok{)\{    }
\NormalTok{                                numer <-}\StringTok{ }\KeywordTok{abs}\NormalTok{(  (b[}\DecValTok{2}\NormalTok{]}\OperatorTok{-}\NormalTok{a[}\DecValTok{2}\NormalTok{])}\OperatorTok{*}\NormalTok{c[}\DecValTok{1}\NormalTok{] }\OperatorTok{-}\NormalTok{(b[}\DecValTok{1}\NormalTok{]}\OperatorTok{-}\NormalTok{a[}\DecValTok{1}\NormalTok{])}\OperatorTok{*}\NormalTok{c[}\DecValTok{2}\NormalTok{] }\OperatorTok{+}\StringTok{ }\NormalTok{b[}\DecValTok{1}\NormalTok{]}\OperatorTok{*}\NormalTok{a[}\DecValTok{2}\NormalTok{] }\OperatorTok{-}\StringTok{ }\NormalTok{b[}\DecValTok{2}\NormalTok{]}\OperatorTok{*}\NormalTok{a[}\DecValTok{1}\NormalTok{])}
\NormalTok{                                denom <-}\StringTok{ }\KeywordTok{sqrt}\NormalTok{( (b[}\DecValTok{2}\NormalTok{]}\OperatorTok{-}\NormalTok{a[}\DecValTok{2}\NormalTok{])}\OperatorTok{^}\DecValTok{2}   \OperatorTok{+}\StringTok{ }\NormalTok{(b[}\DecValTok{1}\NormalTok{]}\OperatorTok{-}\NormalTok{a[}\DecValTok{1}\NormalTok{])}\OperatorTok{^}\DecValTok{2}\NormalTok{ ) }
                                \KeywordTok{return}\NormalTok{(numer}\OperatorTok{/}\NormalTok{denom)  }
\NormalTok{                                 \}}
    
                               \ControlFlowTok{else}\NormalTok{\{}
\NormalTok{                                d1 <-}\StringTok{ }\KeywordTok{dist2}\NormalTok{(a,c)}
\NormalTok{                                d2 <-}\StringTok{ }\KeywordTok{dist2}\NormalTok{(b,c)   }
                                \KeywordTok{return}\NormalTok{( }\KeywordTok{min}\NormalTok{(d1,d2))                              }
\NormalTok{                                 \}                                                }
\NormalTok{                               \}}
    
\NormalTok{    ScaleBeta <-}\StringTok{ }\ControlFlowTok{function}\NormalTok{(X)\{}
\NormalTok{     a<-}\DecValTok{0}
\NormalTok{     b<-pi}
\NormalTok{     y<-X}
\NormalTok{     Samp<-}\DecValTok{50}
    
\NormalTok{     y2 <-}\StringTok{ }\NormalTok{(y}\OperatorTok{-}\NormalTok{a)}\OperatorTok{/}\NormalTok{(b}\OperatorTok{-}\NormalTok{a)}
\NormalTok{     y3<-(y2}\OperatorTok{*}\NormalTok{(Samp }\OperatorTok{-}\StringTok{ }\DecValTok{1}\NormalTok{) }\OperatorTok{+}\StringTok{ }\FloatTok{0.5}\NormalTok{)}\OperatorTok{/}\NormalTok{Samp}
\NormalTok{     y3\}}
    
\NormalTok{    ang<-}\ControlFlowTok{function}\NormalTok{(x2,x1,y2,y1)\{}
\NormalTok{      Theta<-}\KeywordTok{seq}\NormalTok{(}\DecValTok{0}\NormalTok{,}\DecValTok{2}\OperatorTok{*}\NormalTok{pi,}\DataTypeTok{length.out=}\DecValTok{100}\NormalTok{)}
\NormalTok{      DB<-}\KeywordTok{cbind}\NormalTok{(}\KeywordTok{cos}\NormalTok{(Theta),}\KeywordTok{sin}\NormalTok{(Theta))}
\NormalTok{       r <-}\StringTok{ }\KeywordTok{dist}\NormalTok{(x2,x1,y2,y1)}
\NormalTok{       x0<-(x2}\OperatorTok{-}\NormalTok{x1)}\OperatorTok{/}\NormalTok{r}
\NormalTok{       y0<-(y2}\OperatorTok{-}\NormalTok{y1)}\OperatorTok{/}\NormalTok{r}
    
\NormalTok{       (}\KeywordTok{atan2}\NormalTok{(y0,x0) }\OperatorTok{+}\StringTok{ }\NormalTok{pi )}
\NormalTok{       \}}
    
\NormalTok{    ################################################################# Set parameters}
\NormalTok{    Thresh <-}\StringTok{ }\DecValTok{50}         \CommentTok{# Threshold of visual range}
\NormalTok{    Lags<-}\DecValTok{10}             \CommentTok{# Lags at which there are effects of encounters on movement, kinda hardcoded---careful with changes}
\NormalTok{    Steps<-}\DecValTok{150}           \CommentTok{# Length of simulation}
\NormalTok{    Reps <-}\StringTok{ }\DecValTok{30}           \CommentTok{# Number of replications}
    
    \CommentTok{# Adaptive model parameters}
\NormalTok{    Phi0A <-}\StringTok{ }\FloatTok{3.2}
\NormalTok{    Psi0A <-}\StringTok{ }\DecValTok{0}
\NormalTok{    PhiA  <-}\StringTok{ }\KeywordTok{c}\NormalTok{(}\OperatorTok{-}\FloatTok{0.76}\NormalTok{, }\OperatorTok{-}\FloatTok{0.65}\NormalTok{, }\OperatorTok{-}\FloatTok{0.58}\NormalTok{, }\OperatorTok{-}\FloatTok{0.43}\NormalTok{, }\OperatorTok{-}\FloatTok{0.29}\NormalTok{, }\OperatorTok{-}\FloatTok{0.15}\NormalTok{, }\OperatorTok{-}\FloatTok{0.03}\NormalTok{,}\DecValTok{0}\NormalTok{,}\DecValTok{0}\NormalTok{,}\DecValTok{0}\NormalTok{)}
\NormalTok{    PsiA <-}\StringTok{ }\KeywordTok{c}\NormalTok{(}\FloatTok{0.5}\NormalTok{, }\FloatTok{0.3}\NormalTok{, }\FloatTok{0.25}\NormalTok{, }\FloatTok{0.2}\NormalTok{, }\FloatTok{0.17}\NormalTok{, }\FloatTok{0.15}\NormalTok{, }\FloatTok{0.13}\NormalTok{, }\FloatTok{0.11}\NormalTok{, }\FloatTok{0.09}\NormalTok{, }\FloatTok{0.08}\NormalTok{)}
\NormalTok{    OmegaA <-}\StringTok{ }\FloatTok{0.4}
\NormalTok{    EtaA <-}\StringTok{ }\DecValTok{2}
    
     \CommentTok{# Levy parameters}
\NormalTok{    Phi0L <-}\StringTok{ }\FloatTok{2.8}
\NormalTok{    Psi0L <-}\StringTok{ }\DecValTok{0}
\NormalTok{    PhiL  <-}\StringTok{ }\KeywordTok{c}\NormalTok{(}\DecValTok{0}\NormalTok{, }\DecValTok{0}\NormalTok{, }\DecValTok{0}\NormalTok{, }\DecValTok{0}\NormalTok{, }\DecValTok{0}\NormalTok{, }\DecValTok{0}\NormalTok{, }\DecValTok{0}\NormalTok{, }\DecValTok{0}\NormalTok{, }\DecValTok{0}\NormalTok{, }\DecValTok{0}\NormalTok{)}
\NormalTok{    PsiL <-}\StringTok{ }\KeywordTok{c}\NormalTok{(}\DecValTok{0}\NormalTok{, }\DecValTok{0}\NormalTok{, }\DecValTok{0}\NormalTok{, }\DecValTok{0}\NormalTok{, }\DecValTok{0}\NormalTok{, }\DecValTok{0}\NormalTok{, }\DecValTok{0}\NormalTok{, }\DecValTok{0}\NormalTok{, }\DecValTok{0}\NormalTok{, }\DecValTok{0}\NormalTok{)}
\NormalTok{    OmegaL <-}\StringTok{ }\DecValTok{1}
\NormalTok{    EtaL <-}\StringTok{ }\DecValTok{2}
    
     \CommentTok{# Brownian parameters}
\NormalTok{    Phi0B <-}\StringTok{ }\DecValTok{25}
\NormalTok{    Psi0B <-}\StringTok{ }\DecValTok{0}
\NormalTok{    PhiB  <-}\StringTok{ }\KeywordTok{c}\NormalTok{(}\DecValTok{0}\NormalTok{, }\DecValTok{0}\NormalTok{, }\DecValTok{0}\NormalTok{, }\DecValTok{0}\NormalTok{, }\DecValTok{0}\NormalTok{, }\DecValTok{0}\NormalTok{, }\DecValTok{0}\NormalTok{, }\DecValTok{0}\NormalTok{, }\DecValTok{0}\NormalTok{, }\DecValTok{0}\NormalTok{)}
\NormalTok{    PsiB <-}\StringTok{ }\KeywordTok{c}\NormalTok{(}\DecValTok{0}\NormalTok{, }\DecValTok{0}\NormalTok{, }\DecValTok{0}\NormalTok{, }\DecValTok{0}\NormalTok{, }\DecValTok{0}\NormalTok{, }\DecValTok{0}\NormalTok{, }\DecValTok{0}\NormalTok{, }\DecValTok{0}\NormalTok{, }\DecValTok{0}\NormalTok{, }\DecValTok{0}\NormalTok{)}
\NormalTok{    OmegaB <-}\StringTok{ }\DecValTok{15}
\NormalTok{    EtaB <-}\StringTok{ }\DecValTok{2}
\end{Highlighting}
\end{Shaded}

\paragraph{\texorpdfstring{\emph{Modeling bird movement behaviors: step
length, turning angle, spatial location
preference}}{Modeling bird movement behaviors: step length, turning angle, spatial location preference}}\label{modeling-bird-movement-behaviors-step-length-turning-angle-spatial-location-preference}

\begin{Shaded}
\begin{Highlighting}[]
\CommentTok{#This is an R package for modeling bird movement (available at https://github.com/ctross/grackleator)}

\CommentTok{#Install by running on R:}
\KeywordTok{library}\NormalTok{(devtools)}
\KeywordTok{install_github}\NormalTok{(}\StringTok{"Ctross/grackleator"}\NormalTok{)}

\CommentTok{#Some quick examples:}

\CommentTok{#First, lets look at movement dynamics:}

\CommentTok{# Load library and attach data}
\KeywordTok{library}\NormalTok{(grackleator)  }

\CommentTok{# Simulate tracks from 1 grackle over 30 trips}
\NormalTok{z <-}\StringTok{ }\KeywordTok{grackleate}\NormalTok{(}\DataTypeTok{AlphaDist=}\FloatTok{3.8}\NormalTok{,}\DataTypeTok{AlphaAngle=}\DecValTok{0}\NormalTok{,}\DataTypeTok{SDDist=}\FloatTok{1.5}\NormalTok{,}\DataTypeTok{DAngle=}\DecValTok{2}\NormalTok{)}

\CommentTok{# Analyze results of trips for step-size and angle change}
\NormalTok{m1 <-}\StringTok{ }\KeywordTok{grackleize}\NormalTok{(z}\OperatorTok{$}\NormalTok{X,z}\OperatorTok{$}\NormalTok{Y)}
 
\CommentTok{# Plot results over trips}
\KeywordTok{grackletrip}\NormalTok{(m1)}

\CommentTok{# Plot bird-specific parameters}
\KeywordTok{gracklepars}\NormalTok{(m1)}

\CommentTok{#And now habitat selection:}
\CommentTok{# Load library and attach data}
\KeywordTok{library}\NormalTok{(grackleator)  }

\CommentTok{# Simulate tracks from 1 grackle over 30 trips}
\NormalTok{z <-}\StringTok{ }\KeywordTok{grackleate}\NormalTok{(}\DataTypeTok{AlphaDist=}\FloatTok{3.8}\NormalTok{,}\DataTypeTok{AlphaAngle=}\DecValTok{0}\NormalTok{,}\DataTypeTok{SDDist=}\FloatTok{1.5}\NormalTok{,}\DataTypeTok{DAngle=}\DecValTok{2}\NormalTok{)}

\CommentTok{# Bin data on 2D grid}
\NormalTok{GrackBins<-}\StringTok{ }\KeywordTok{gracklebinner}\NormalTok{(z,}\DataTypeTok{ab_override=}\KeywordTok{matrix}\NormalTok{(}\KeywordTok{c}\NormalTok{(}\OperatorTok{-}\DecValTok{3000}\NormalTok{,}\OperatorTok{-}\DecValTok{3000}\NormalTok{,}\DecValTok{3000}\NormalTok{,}\DecValTok{3000}\NormalTok{),}\DataTypeTok{nrow=}\DecValTok{2}\NormalTok{,}\DataTypeTok{ncol=}\DecValTok{2}\NormalTok{))}

\CommentTok{# Model location data}
\NormalTok{m2 <-}\StringTok{ }\KeywordTok{gracklenomial}\NormalTok{(GrackBins)}
 
\CommentTok{# Plot environmental hotspots}
\KeywordTok{image}\NormalTok{(}\KeywordTok{matrix}\NormalTok{(}\KeywordTok{colSums}\NormalTok{(GrackBins), }\DataTypeTok{nrow=}\DecValTok{15}\NormalTok{,}\DataTypeTok{ncol=}\DecValTok{15}\NormalTok{)) }\CommentTok{# Overall}
\KeywordTok{image}\NormalTok{(}\KeywordTok{matrix}\NormalTok{(GrackBins[}\DecValTok{1}\NormalTok{,], }\DataTypeTok{nrow=}\DecValTok{15}\NormalTok{,}\DataTypeTok{ncol=}\DecValTok{15}\NormalTok{)) }\CommentTok{# Day 1}
\KeywordTok{image}\NormalTok{(}\KeywordTok{matrix}\NormalTok{(GrackBins[}\DecValTok{2}\NormalTok{,], }\DataTypeTok{nrow=}\DecValTok{15}\NormalTok{,}\DataTypeTok{ncol=}\DecValTok{15}\NormalTok{)) }\CommentTok{# Day 2}
\KeywordTok{image}\NormalTok{(}\KeywordTok{matrix}\NormalTok{(GrackBins[}\DecValTok{3}\NormalTok{,], }\DataTypeTok{nrow=}\DecValTok{15}\NormalTok{,}\DataTypeTok{ncol=}\DecValTok{15}\NormalTok{)) }\CommentTok{# Day 3}
\KeywordTok{image}\NormalTok{(}\KeywordTok{matrix}\NormalTok{(}\KeywordTok{get_posterior_mean}\NormalTok{(m2,}\DataTypeTok{pars=}\StringTok{"A"}\NormalTok{), }\DataTypeTok{nrow=}\DecValTok{15}\NormalTok{,}\DataTypeTok{ncol=}\DecValTok{15}\NormalTok{)) }\CommentTok{# Overall}

\CommentTok{# Plot bird-specific parameters}
\KeywordTok{grackleations}\NormalTok{(m2)}

\CommentTok{# Now create data with temporal correlations}
\NormalTok{GrackBins2 <-}\StringTok{ }\NormalTok{GrackBins}
\NormalTok{A <-}\StringTok{ }\KeywordTok{get_posterior_mean}\NormalTok{(m2,}\DataTypeTok{pars=}\StringTok{"A"}\NormalTok{)}
\NormalTok{A <-}\StringTok{ }\NormalTok{A}\OperatorTok{/}\KeywordTok{sum}\NormalTok{(A)}
\NormalTok{B <-}\StringTok{ }\FloatTok{0.8}

\ControlFlowTok{for}\NormalTok{(i }\ControlFlowTok{in} \DecValTok{2}\OperatorTok{:}\DecValTok{30}\NormalTok{)}
\NormalTok{GrackBins2[i,] <-}\StringTok{ }\KeywordTok{rmultinom}\NormalTok{(}\DecValTok{1}\NormalTok{,}\KeywordTok{sum}\NormalTok{(GrackBins2[i}\OperatorTok{-}\DecValTok{1}\NormalTok{,]),A}\OperatorTok{*}\NormalTok{(}\DecValTok{1}\OperatorTok{-}\NormalTok{B) }\OperatorTok{+}\StringTok{ }\NormalTok{B}\OperatorTok{*}\NormalTok{(GrackBins2[i}\OperatorTok{-}\DecValTok{1}\NormalTok{,]}\OperatorTok{/}\KeywordTok{sum}\NormalTok{(GrackBins2[i}\OperatorTok{-}\DecValTok{1}\NormalTok{,])))}

\CommentTok{# Analyze the new data}
\NormalTok{m3 <-}\StringTok{ }\KeywordTok{gracklenomial}\NormalTok{(GrackBins2)}

\CommentTok{# Plot environmental hotspots}
\KeywordTok{image}\NormalTok{(}\KeywordTok{matrix}\NormalTok{(}\KeywordTok{colSums}\NormalTok{(GrackBins2), }\DataTypeTok{nrow=}\DecValTok{15}\NormalTok{,}\DataTypeTok{ncol=}\DecValTok{15}\NormalTok{)) }\CommentTok{# Overall}
\KeywordTok{image}\NormalTok{(}\KeywordTok{matrix}\NormalTok{(GrackBins2[}\DecValTok{1}\NormalTok{,], }\DataTypeTok{nrow=}\DecValTok{15}\NormalTok{,}\DataTypeTok{ncol=}\DecValTok{15}\NormalTok{)) }\CommentTok{# Day 1}
\KeywordTok{image}\NormalTok{(}\KeywordTok{matrix}\NormalTok{(GrackBins2[}\DecValTok{2}\NormalTok{,], }\DataTypeTok{nrow=}\DecValTok{15}\NormalTok{,}\DataTypeTok{ncol=}\DecValTok{15}\NormalTok{)) }\CommentTok{# Day 2}
\KeywordTok{image}\NormalTok{(}\KeywordTok{matrix}\NormalTok{(GrackBins2[}\DecValTok{3}\NormalTok{,], }\DataTypeTok{nrow=}\DecValTok{15}\NormalTok{,}\DataTypeTok{ncol=}\DecValTok{15}\NormalTok{)) }\CommentTok{# Day 3}

\KeywordTok{image}\NormalTok{(}\KeywordTok{matrix}\NormalTok{(}\KeywordTok{get_posterior_mean}\NormalTok{(m3,}\DataTypeTok{pars=}\StringTok{"A"}\NormalTok{), }\DataTypeTok{nrow=}\DecValTok{15}\NormalTok{,}\DataTypeTok{ncol=}\DecValTok{15}\NormalTok{)) }\CommentTok{# Post}

\CommentTok{# Plot bird-specific parameters}
\KeywordTok{grackleations}\NormalTok{(m3)}
\end{Highlighting}
\end{Shaded}

\paragraph{\texorpdfstring{\emph{H1: P1 - Exploration measured in
captivity relates to space use
behavior}}{H1: P1 - Exploration measured in captivity relates to space use behavior}}\label{h1-p1---exploration-measured-in-captivity-relates-to-space-use-behavior}

\begin{Shaded}
\begin{Highlighting}[]
\NormalTok{data <-}\StringTok{ }\KeywordTok{read.csv}\NormalTok{(}\StringTok{"Space_use.csv"}\NormalTok{, }\DataTypeTok{header =}\NormalTok{ T)}

\CommentTok{# Home range }
\NormalTok{m1 =}\StringTok{ }\KeywordTok{lm}\NormalTok{(}\KeywordTok{log}\NormalTok{(area) }\OperatorTok{~}\StringTok{ }\NormalTok{ExpObj }\OperatorTok{+}\StringTok{ }\NormalTok{ExpEnv }\OperatorTok{+}\StringTok{ }\NormalTok{Sex }\OperatorTok{+}\StringTok{ }\NormalTok{Condition, }\DataTypeTok{data =}\NormalTok{ data)}
\KeywordTok{hist}\NormalTok{(m1}\OperatorTok{$}\NormalTok{resid)}
\KeywordTok{summary}\NormalTok{(m1)}

\CommentTok{# Step length }
\NormalTok{m2 =}\StringTok{ }\KeywordTok{lm}\NormalTok{(}\KeywordTok{log}\NormalTok{(std_step)) }\OperatorTok{~}\StringTok{ }\NormalTok{ExpObj }\OperatorTok{+}\StringTok{ }\NormalTok{ExpEnv }\OperatorTok{+}\StringTok{ }\NormalTok{Sex }\OperatorTok{+}\StringTok{ }\NormalTok{Condition, data =}\StringTok{ }\NormalTok{data}\ErrorTok{)}
\KeywordTok{hist}\NormalTok{(m2}\OperatorTok{$}\NormalTok{resid)}
\KeywordTok{summary}\NormalTok{(m2)}

\CommentTok{# Turning angle}
\NormalTok{m3 =}\StringTok{ }\KeywordTok{lm}\NormalTok{(}\KeywordTok{log}\NormalTok{(std_angle)) }\OperatorTok{~}\StringTok{ }\NormalTok{ExpObj }\OperatorTok{+}\StringTok{ }\NormalTok{ExpEnv }\OperatorTok{+}\StringTok{ }\NormalTok{Sex }\OperatorTok{+}\StringTok{ }\NormalTok{Condition, data =}\StringTok{ }\NormalTok{data}\ErrorTok{)}
\KeywordTok{hist}\NormalTok{(H3}\OperatorTok{$}\NormalTok{resid)}
\KeywordTok{summary}\NormalTok{(m3)}

\CommentTok{# Spatial preferences}
\NormalTok{m4 =}\StringTok{ }\KeywordTok{lm}\NormalTok{(}\KeywordTok{log}\NormalTok{(loc_pref)) }\OperatorTok{~}\StringTok{ }\NormalTok{ExpObj }\OperatorTok{+}\StringTok{ }\NormalTok{ExpEnv }\OperatorTok{+}\StringTok{ }\NormalTok{Sex }\OperatorTok{+}\StringTok{ }\NormalTok{Condition, data =}\StringTok{ }\NormalTok{data}\ErrorTok{)}
\KeywordTok{hist}\NormalTok{(H4}\OperatorTok{$}\NormalTok{resid)}
\KeywordTok{summary}\NormalTok{(m4)}
\end{Highlighting}
\end{Shaded}

\paragraph{\texorpdfstring{\emph{H2: P2 - Space use behaviors vary among
populations across the
range}}{H2: P2 - Space use behaviors vary among populations across the range}}\label{h2-p2---space-use-behaviors-vary-among-populations-across-the-range}

\begin{Shaded}
\begin{Highlighting}[]
\NormalTok{data <-}\StringTok{ }\KeywordTok{read.csv}\NormalTok{(}\StringTok{"Space_use.csv"}\NormalTok{, }\DataTypeTok{header =}\NormalTok{ T)}

\CommentTok{# Home range }
\NormalTok{m1 =}\StringTok{ }\KeywordTok{lm}\NormalTok{(}\KeywordTok{log}\NormalTok{(area) }\OperatorTok{~}\StringTok{ }\NormalTok{Site }\OperatorTok{+}\StringTok{ }\NormalTok{Sex }\OperatorTok{+}\StringTok{ }\NormalTok{Condition, }\DataTypeTok{data =}\NormalTok{ data)}
\KeywordTok{hist}\NormalTok{(m1}\OperatorTok{$}\NormalTok{resid)}
\KeywordTok{summary}\NormalTok{(m1)}

\CommentTok{# Step length}
\NormalTok{m2 =}\StringTok{ }\KeywordTok{lm}\NormalTok{(}\KeywordTok{log}\NormalTok{(std_step)) }\OperatorTok{~}\StringTok{ }\NormalTok{Site }\OperatorTok{+}\StringTok{ }\NormalTok{Sex }\OperatorTok{+}\StringTok{ }\NormalTok{Condition, data =}\StringTok{ }\NormalTok{data}\ErrorTok{)}
\KeywordTok{hist}\NormalTok{(m2}\OperatorTok{$}\NormalTok{resid)}
\KeywordTok{summary}\NormalTok{(m2)}

\CommentTok{# Turning angle}
\NormalTok{m3 =}\StringTok{ }\KeywordTok{lm}\NormalTok{(}\KeywordTok{log}\NormalTok{(std_angle)) }\OperatorTok{~}\StringTok{ }\NormalTok{Site }\OperatorTok{+}\StringTok{ }\NormalTok{Sex }\OperatorTok{+}\StringTok{ }\NormalTok{Condition, data =}\StringTok{ }\NormalTok{data}\ErrorTok{)}
\KeywordTok{hist}\NormalTok{(m3}\OperatorTok{$}\NormalTok{resid)}
\KeywordTok{summary}\NormalTok{(m3)}

\CommentTok{# Spatial preference}
\NormalTok{m4 =}\StringTok{ }\KeywordTok{lm}\NormalTok{(}\KeywordTok{log}\NormalTok{(loc_pref)) }\OperatorTok{~}\StringTok{ }\NormalTok{Site }\OperatorTok{+}\StringTok{ }\NormalTok{Sex }\OperatorTok{+}\StringTok{ }\NormalTok{Condition, data =}\StringTok{ }\NormalTok{data}\ErrorTok{)}
\KeywordTok{hist}\NormalTok{(m4}\OperatorTok{$}\NormalTok{resid)}
\KeywordTok{summary}\NormalTok{(m4)}
\end{Highlighting}
\end{Shaded}

\subsubsection{E. ETHICS}\label{e.-ethics}

This research is carried out in accordance with permits from the:

\begin{enumerate}
\def\labelenumi{\arabic{enumi})}
\tightlist
\item
  US Fish and Wildlife Service (scientific collecting permit number
  MB76700A-0,1,2)
\item
  US Geological Survey Bird Banding Laboratory (federal bird banding
  permit number 23872)
\item
  Arizona Game and Fish Department (scientific collecting license number
  SP594338 {[}2017{]}, SP606267 {[}2018{]}, and SP639866 {[}2019{]})
\item
  Institutional Animal Care and Use Committee at Arizona State
  University (protocol number 17-1594R)
\end{enumerate}

\subsubsection{F. AUTHOR CONTRIBUTIONS}\label{f.-author-contributions}

\textbf{McCune:} Hypothesis development, data collection, data analysis
and interpretation, write up, revising/editing.

\textbf{Folsom:} Data collection, revising/editing.

\textbf{Ross:} Model development, data analysis and interpretation,
revising/editing.

\textbf{Bergeron:} Data collection, revising/editing.

\textbf{Logan:} Hypothesis development, data interpretation, write up,
revising/editing, materials/funding.

\subsubsection{G. FUNDING}\label{g.-funding}

This research is funded by the Department of Human Behavior, Ecology and
Culture at the Max Planck Institute for Evolutionary Anthropology.

\subsubsection{I. CONFLICT OF INTEREST
DISCLOSURE}\label{i.-conflict-of-interest-disclosure}

We, the authors, declare that we have no financial conflicts of interest
with the content of this article. Corina Logan is a Recommender and on
the Managing Board at PCI Ecology.

\subsubsection{J. ACKNOWLEDGEMENTS}\label{j.-acknowledgements}

We thank Melissa Wilson Sayres for sponsoring our affiliations at
Arizona State University (ASU); Kevin Langergraber for serving as local
PI on the ASU IACUC; Kristine Johnson for technical advice on
great-tailed grackles; Julia Cissewski for tirelessly solving problems
involving financial transactions and contracts; Richard McElreath for
project support; and our research assistants: Aldora Messinger, Elysia
Mamola, Michael Guillen, Rita Barakat, Adriana Boderash, Olateju
Ojekunle, August Sevchik, Justin Huynh, Jennifer Berens, Michael Pickett
and Amanda Overholt.

\subsubsection{\texorpdfstring{K
\href{MyLibrary.bib}{REFERENCES}}{K REFERENCES}}\label{k-references}


\end{document}
